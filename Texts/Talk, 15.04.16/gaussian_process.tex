\begin{frame}{Gaussian Process}
			\begin{definition}
				A Gaussian process is a collection of random variables, any finite number of which have a joint Gaussian distribution.
			\end{definition}

			$$f \sim \GP(m(\cdot), k(\cdot, \cdot)) \Leftrightarrow \forall t_1, \ldots, t_n \hspace{0.5cm} (f(t_1), \ldots, f(t_n)) \sim \N(\mu, K),$$
			where $\mu = (m(t_1), \ldots, m(t_n))^T$, $K \in \R^{n \times n}$, $K_{ij} = k(t_i, t_j)$.

			$m: \R \rightarrow \R$ is called the mean function of the gaussian process $f$.

			$k: \R \times \R \rightarrow \R_+$ is the covariance function of $f$.

			\vspace{0.5cm}
			Mean and covariance functions completely determine a gaussian process.

		\end{frame}

		\begin{frame}{Example}
			\begin{figure}[!h]
				\centering
				\subfloat{
					\scalebox{0.5}{
						\input{../../Code/Experiments/pictures/1dgp-regression_nodata.pgf}
					}
				}
				\subfloat{
					\scalebox{0.5}{
						\input{../../Code/Experiments/pictures/2dgp-regression_nodata.pgf}
					}
				}
				\caption{Examples of gaussian processes}
			\end{figure}
		\end{frame}

		\begin{frame}{Covariance functions}
			\begin{figure}[!h]
				\centering

				\subfloat{
					\scalebox{0.3}{
						\input{../../Code/Experiments/pictures/1dgp-regression_se_01.pgf}
					}
				}
				\subfloat{
					\scalebox{0.3}{
						\input{../../Code/Experiments/pictures/1dgp-regression_se_05.pgf}
					}
				}
				\subfloat{
					\scalebox{0.3}{
						\input{../../Code/Experiments/pictures/1dgp-regression_se_1.pgf}
					}
				}

				\subfloat{
					\scalebox{0.3}{
						\input{../../Code/Experiments/pictures/1dgp-regression_matern_01.pgf}
					}
				}
				\subfloat{
					\scalebox{0.3}{
						\input{../../Code/Experiments/pictures/1dgp-regression_matern_05.pgf}
					}
				}
				\subfloat{
					\scalebox{0.3}{
						\input{../../Code/Experiments/pictures/1dgp-regression_matern_1.pgf}
					}
				}
				\caption{Squared-exponential and Matern covariance functions}
			\end{figure}
		\end{frame}