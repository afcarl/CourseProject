\begin{frame}{Notation}
	\begin{itemize}
		\item $\{(x_i, f_i) | i = 1, \ldots, n\}$ — dataset, considered to be generated from a Gaussian process $f \sim \GP(m(\cdot), k(\cdot, \cdot))$, let $x \in \R^d$. 
		\item $X \in \R^{n \times d}$ — the matrix, comprised of data points $x_1, \ldots, x_n$.
		\item $f \in \R^n$ — the vector of target values $f_1, \ldots, f_n$.

		\item $y \in \R^n$ — the noisy version of $f$: $y \sim \N(y | f, \sigma_n^2 I)$

		\item $X_* \in \R^{k \times d}$ — new (test) data points.

		\item $f_* \in \R^k$ — the desired vector of process values at new data points $X_*$.

		\item $K(X, X) \in \R^{n \times n}$ — the matrix, comprised of pairvise values of the covariance function $k(\cdot, \cdot)$ of the underlying process:
		$$K(X, X)_{ij} = k(x_i, x_j).$$

		\item $K(X, X_*) \in \R^{n \times k}$ — the matrix, defined similarly to the $K(X, X)$.

		\item $K(X_*, X) = K(X, X_*)^T$.
	\end{itemize}

	% \vspace{0.3cm}
	% We want to obtain the predictive distribution
	
\end{frame}

\begin{frame}{Problem statement}
	The model is as follows. $f$ is a vector of unobserved samples from a gaussian process $\GP(m(\cdot), k(\cdot, \cdot))$ at data points $x_i$ and $y$ is it's noisy version we observe.
	$$f \sim \N(m(X), K(X, X)),$$
	$$y = f + \varepsilon, \hspace{0.5cm}\varepsilon \sim \N(0, \sigma_n^2 I).$$
	What we want to obtain is the predictive distribution at a set of new data points $X_*$
	$$p(f_* | X_*, X, y).$$

	We put the following prior on our model: the data is generated from a zero-mean gaussian process with covariance function $k(\cdot, \cdot)$:

	$$f \sim \GP(0, k(\cdot, \cdot)).$$

	This prior is not limiting, because zero-mean prior does not imply a zero-mean predictive distribution.

\end{frame}

% \begin{frame}{Noise-free case}
% 	Let us consider the noise-free case first:
% 	$$y = f.$$

% 	

	

% \end{frame}

% \begin{frame}{Noise-free case}

% 	As $f$ and $f_{*}$ are assumed to be generated from the same gaussian proces, we have the following joint distribution

% 	$$\left [ \begin{array}{c} f\\ f_* \end{array} \right ]
% 	\sim
% 	\N \left ( 0, \left [\begin{array}{cc} K(X, X) & K(X, X_*)\\ K(X_*, X) & K(X_*, X_*) \end{array} \right] \right ).
% 	$$

% 	Marginalizing this distribution, we obtain the predictive
% 	% $$f_* | X_*, X, f $$
% 	$$f_* | X_*, X, f \sim \N( \hat m, \hat K ),$$
% 	where 
% 	$$\E [f_* | f ] = \hat m = K(X_*, X) K(X, X)^{-1} f,$$
% 	$$\cov(f_* | f ) = \hat K = K(X_*, X_*) - K(X_*, X)K(X, X)^{-1}K(X, X_*).$$

% 	% \hspace{0.5cm}
% 	% The computational complexity of this method is defined by the complexity of inversing the $K(X, X)$ matrix, and thus the method scales as $O(N^3)$.
% \end{frame}

\begin{frame}{GP-Regression}
	% In the noisy case, we have a slightly different model:
	% $$y = f + \varepsilon,$$
	% where $\varepsilon \sim \N(0, \sigma_n^2 I)$.

	The joint distribution for the process values $f$ and $f_*$ is given by
	$$\left [ \begin{array}{c} f\\ f_* \end{array} \right ]
	\sim
	\N \left ( 0, \left [\begin{array}{cc} K(X, X) & K(X, X_*)\\ K(X_*, X) & K(X_*, X_*) \end{array} \right] \right ).
	$$

	We slightly change the covariance matrix to obtain the joint distribution for $y$ and $f_*$.
	$$\left [ \begin{array}{c} y\\ f_* \end{array} \right ]
	\sim
	\N \left ( 0, \left [\begin{array}{cc} K(X, X) + \sigma_n^2 I & K(X, X_*)\\ K(X_*, X) & K(X_*, X_*) \end{array} \right] \right ).
	$$
\end{frame}

\begin{frame}{Predictive distribution}
	Conditioning the joint distribution, we obtain the predictive
	$$f_* | y \sim \N( \hat m, \hat K ),$$
	$$\E[f_* | y] = \hat m = K(X_*, X) (K(X, X) + \sigma_n^2 I)^{-1} y,$$
	$$\cov(f_* | y ) = \hat K = K(X_*, X_*) - K(X_*, X)(K(X, X) + \sigma_n^2 I)^{-1}K(X, X_*).$$
\end{frame}

\begin{frame}{Example}
	\begin{figure}[!h]
		\centering
		\subfloat{
			\scalebox{0.5}{
				\input{../../Code/Experiments/pictures/1dgp-regression_noopt.pgf}
			}
		}
	\end{figure}

	As we can see, the model does not explain the data very well. In order to deal with this problem, we can tweak the covariance function. The covariance functions usualy have a set of parameters, which we will refer to as covariance (or kernel) hyper-parameters. Varying these parameters, we can find a better model for the data.

\end{frame}

\begin{frame}{Marginal likelihood}
	In order to find the best set of kernel hyper-parameters, we maximize the marginal likelihood with respect to them. In the case of gaussian process regression, this likelihood is given by
	$$\log p(y) = \log \int p(y | f) p(f) df = \log \N(y| 0, K(X, X) + \sigma_n^2 I) = $$
	$$ = -\frac 1 2 y^{T} (K(X, X) + \sigma_n^2 I)^{-1} y - \frac 1 2 \log |K(X, X) + \sigma_n^2 I| - \frac n 2 \log 2 \pi.$$

	If $k(\cdot, \cdot)$ is a differentiable function of it's hyper-parameters (which is usually true), $p(y)$ also is, and can be maximized with gradient-based optimization methods.
\end{frame}

\begin{frame}{Example}

	\begin{figure}[!h]
		\centering
		\subfloat{
			\scalebox{0.5}{
				\input{../../Code/Experiments/pictures/1dgp-regression_noopt.pgf}
			}
		}
		\subfloat{
			\scalebox{0.5}{
				\input{../../Code/Experiments/pictures/1dgp-regression_opt.pgf}
			}
		}
		\caption{Predictive distribution before and after hyper-parameter adaptation}
	\end{figure}

	As we can see, after adaptation of kernel hyper-parameters, the method does a better job, explaining the data.
\end{frame}

\begin{frame}{Computational Complexity}
	
	The predictive mean and covariance are given by
	$$\E[f_* | y] = \hat m = K(X_*, X) (K(X, X) + \sigma_n^2 I)^{-1} y,$$
	$$\cov(f_* | y ) = \hat K = K(X_*, X_*) - K(X_*, X)(K(X, X) + \sigma_n^2 I)^{-1}K(X, X_*).$$

	Marginal likelihood is given by
	$$\log p(y) = -\frac 1 2 y^{T} (K(X, X) + \sigma_n^2 I)^{-1} y - \frac 1 2 \log |K(X, X) + \sigma_n^2 I| - \frac n 2 \log 2 \pi.$$

	The computational complexity of the gaussian process regression is determined by the complexity of inversing the $K(X, X)$ matrix and computing the detrminant of $K(X, X) + \sigma_n^2 I$, and thus scales as $O(n^3)$. This complexity makes the method inapplicable to big problems and thus approximate approaches are needed.
\end{frame}