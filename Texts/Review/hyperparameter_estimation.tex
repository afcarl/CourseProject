Bayesian paradigm provides a way of estimating the kernel hyper-parameters of the GP-model through maximizization of the marginal likelihood of the model. Marginal likelihood is given by
	$$p(y | X) = \int p(y | f, X) p(f | X) df,$$
	which is the likelihood, marginalized over the hidden values $f$ of the underlying process.

	For the GP-regression the marginal likelihood can be computed in claused form and is given by
	\begin{equation}
		\label{regression_ml}
		\log p(y | X) = -\frac 1 2 y^{T} (K + \sigma_n^2 I)^{-1} y - \frac 1 2 \log |K + \sigma_n^2 I| - \frac n 2 \log 2 \pi.
	\end{equation}

	For the method, described in section \ref{gp-classification} the marginal likelihood can be computed as follows.
	$$p(y | X) = \int p(y | f, X) p(f | X) df = \int \exp{\Psi(f)} df,$$
	where we use the notation from section \ref{gp-classification}. Using the Taylor expansion, locally near $\hat f$ we have $\Psi(f) \simeq \Psi(\hat f) + \frac 1 2(f - \hat f)^T A (f - \hat f)$, where $A$ is the hessian of $\Psi$ at $\hat f$. Using this approximation we obtain
	$$p(y | X) \simeq q(y | X) = \exp(\Psi(\hat f)) \int \exp( - \frac 1 2 (f - \hat f)^T A (f - \hat f)) df.$$
	This last integral can be calculated analytically to obtain a closed form approximation to the log marginal likelihood. 
	\begin{equation}
		\label{classification_ml}
		\log q(y|X) = -\frac 1 2 \hat f^T K^{-1} \hat f + \log p(y|\hat f) - \frac 1 2 \log|B|,
	\end{equation}
	where 
	$$|B| = |K| \left|- \left. \frac{\partial^2 \log p(y | f)}{\partial f^2} \right|_{f = \hat f} \right|.$$ 

	Using the derived formulas \ref{classification_ml} and \ref{regression_ml} we can find the optimal values of hyper-parameters through maximization of the marginal likelihood of the corresponding model.