\documentclass[12pt]{article}

\usepackage[utf8]{inputenc}
\usepackage[russian]{babel}
\usepackage[T2A]{fontenc}
\usepackage{textcomp}
\usepackage{a4wide}
\usepackage{amsmath, amssymb}
\usepackage{graphicx}
\usepackage{wrapfig}
\usepackage{caption}
\usepackage{subfig}
\usepackage{listings}
\usepackage{hyperref}
% \usepackage{fontspec}
\usepackage{pgfplots}
\usepackage{tikz}
\usepackage{amsthm}
\usepackage{pgf,pgfarrows,pgfnodes}
\usepackage{pgf}

\lstset{
language=Python,
basicstyle=\ttfamily\small,
otherkeywords={self},                   
}

\title{Title}
\title{Неточный метод Ньютона.}
\date{4 октября 2015}
\author{Павел Измаилов}

\begin{document}

\renewcommand{\contentsname}{\centerline{\bf Contents}}
\renewcommand{\refname}{\centerline{\bf Literature}}

\newcommand{\GP}{\mathcal{GP}}
\newcommand{\E}{\mathbb{E}}
\newcommand{\R}{\mathbb{R}}
\newcommand{\N}{\mathcal{N}}
\newcommand{\cov}{\mbox{cov}}
\newcommand{\Nystrom}{Nystr\"{o}m }
\newcommand{\KL}[2]{\mbox{KL}\left(#1\mbox{ || }#2\right)}
\newcommand{\tr}{\mbox{tr}}
\newcommand{\derivative}[2]{\frac{\partial #1}{\partial #2}}
\newcommand{\sndderivative}[3]{\frac{\partial^2 #1}{\partial #2 \partial #3}}
\newcommand{\bigO}{\mathcal{O}}

\newlength{\arrayrulewidthOriginal}
\newcommand{\Cline}[2]{%
  \noalign{\global\setlength{\arrayrulewidthOriginal}{\arrayrulewidth}}%
  \noalign{\global\setlength{\arrayrulewidth}{#1}}\cline{#2}%
  \noalign{\global\setlength{\arrayrulewidth}{\arrayrulewidthOriginal}}}

\newtheorem{definition}{Definition}
\newtheorem{theorem}{Theorem}


\def\vec#1{\mathchoice{\mbox{\boldmath$\displaystyle#1$}}
{\mbox{\boldmath$\textstyle#1$}} {\mbox{\boldmath$\scriptstyle#1$}} {\mbox{\boldmath$\scriptscriptstyle#1$}}}

%\maketitle
\centerline{Lomonosov Moscow State University}

\centerline{Faculty of Computer Science}

\vspace{5 cm}

\centerline{\Large Review of matherials on}

\vspace{1 cm}

\centerline{\Large \bf Gaussian Processes for Machine Learning}

\vspace{6 cm}

\begin{flushright}

Pavel Izmailov
\end{flushright}

\vfill 

\centerline{Moscow,  2016}
\thispagestyle{empty} 
\pagebreak

\section{Theory}

In this section an introduction to Gaussian process theory is provided.

\subsection{Gaussian Process}
	Consider the following definition
	\begin{definition}
		A Gaussian process is a collection of random variables, any finite number of which have a joint Gaussian distribution.
	\end{definition}
	A Gaussian process is completely specified by it's mean function and covariance function. These functions are defined as follows
	\begin{definition}
		Let $f(x)$ be a real-valued Gaussian process. Then the functions
		$$m(x) = \E[f(x)],$$
		$$k(x, x') = \E[(f(x) - m(x)) (f(x') - m(x'))],$$
		are the mean function and the covariance function of the process $f$ respectively. 
	\end{definition}
	
	We will write the Gaussian process as $f(x) \sim \GP(m(x), k(x, x'))$.
	
\subsection{GP-regression}
	Consider the following task. We have a dataset $\{(x_i, f_i) | i = 1, \ldots, n\}$, generated from a Gaussian process $f \sim \GP(m(x), k(x, x'))$, let $x \in \R^d$.  We will denote the matrix comprised of points $x_1, \ldots, x_n$ by $X \in \R^{n \times d}$ and the vector of corresponding values $f_1, ..., f_n$ by $f \in \R^n$. We want to predict the values $f_* \in \R^m$ of this random process at a set of other m points $X_* \in \R^{m \times d}$. The joint distribution of $f$ and $f_*$ is given by
	$$
	\left [ \begin{array}{c} f\\ f_* \end{array} \right ]
	\sim
	\N \left ( 0, \left [\begin{array}{cc} K(X, X) & K(X, X_*)\\ K(X_*, X) & K(X_*, X_*) \end{array} \right] \right ),
	$$
	where $K(X, X) \in \R^{n \times n}$, $K(X, X_*) = K(X^*, X)^T \in \R^{n \times m}$, $K(X^*, X^*) \in \R^{m \times m}$ are the matrices comprised of pairwise values of the covariance function $k$ for the given sets.
	
	The conditional distribution
	
	$$f_* | X_*, X, f \sim \N( \hat m, \hat K ),$$
	where 
	$$\E [f_* | f ] = \hat m = K(X_*, X) K(X, X)^{-1} f,$$
	$$\cov(f_* | f ) = \hat K = K(X_*, X_*) - K(X_*, X)K(X, X)^{-1}K(X, X_*).$$
		
	Thus, predicting the values of the Gaussian process at a new data point requires solving a linear system with a matrix of size $n \times n$ and thus scales as $O(n^3)$.

	\begin{figure}[!h]
		\centering
		\subfloat{
			\scalebox{0.7}{
				\input{../../Code/Experiments/pictures/1dgp-regression.pgf}
			}
		}
		\subfloat{
			\scalebox{0.7}{
	    		\input{../../Code/Experiments/pictures/2dgp-regression.pgf}
			}
		}
		\caption{One and two-dimensional gaussian processes}
		\label{brute_reg_example}
	\end{figure}


	In fig. \ref{brute_reg_example} you can see the examples of one and two-dimensional gaussian-processes, reconstructed from the data. The data points are shown by black `$+$' signs.
	
	\subsubsection{Noisy case}
		Consider the following model. We now have a dataset $\{(x_i, y_i)| i = 1, \ldots n\}$, where $y_i = f(x_i) + \varepsilon$, $\varepsilon \sim \N(0, \sigma_n)$. This means that we only have access to the noisy observations and not the true values of the process at data points. With the notation and logics similar to the one we used it the previous section we can find the conditional distribution for the values $f_*$ of the process at new points $X_*$ in this case:
		$$f_* | y \sim \N( \hat m, \hat K ),$$
		$$\E[f_* | y] = \hat m = K(X_*, X) (K(X, X) + \sigma_n^2 I)^{-1} y,$$
		$$\cov(f_* | y ) = \hat K = K(X_*, X_*) - K(X_*, X)(K(X, X) + \sigma_n^2 I)^{-1}K(X, X_*).$$
		
\subsection{GP-classification}
	\label{gp-classification}
Another important class of problems in machine learning is classification. We will consider the following problem. We have a dataset $\{(x_i, y_i) | i = 1, \ldots, n\}$, where $x_i \in \R^d$, $y_i \in \{-1, 1\}$. We want to predict the probabilities of new datapoints $x_*$ belonging to positive class.

We will consider the following model. We will introduce a latent function $f: \R^d \rightarrow \R$ and put a zero-mean GP prior over it. 
$$f \sim \GP(0, k(\cdot, \cdot)).$$
We will then consider the probability of the object $x_*$ belonging to positiva class, to be equal to $\sigma(f(x_*))$ for the chosen sigmoid function $\sigma$.
$$p(y_* = +1 | x_*) = \sigma(f(x_*)).$$

Note, that the graphical for this model is exactly the same, as for regression problem and is given in fig. \ref{gp_graphical_model}.

We will use the logistic function $\sigma(z) = (1 + \exp(-z))^{-1}$, however one can use other sigmoid functions as well.

Now inference can be done in two steps. First, for the new data point $x_*$ we should find the conditional distribution of the value of the latent process $f$ at the new data point $x_*$. This can be computed as follows
\begin{equation}
	\label{classification_conditional}
	p(f_* | X, y, x_*) = \int p(f_* | X, x_*, f) p(f | X, y) df.
\end{equation}
Now, the probability of the positive class is given by marginalizing over the latent variable $f_*$.
\begin{equation}
	\label{classification_class_probability}
	p(y_* = +1 | X, y, x_*) = \int \sigma(f_*) p(f_* | X, y, x_*) df_*.
\end{equation}

Unfortunantely, both the integrals in (\ref{classification_conditional}) and (\ref{classification_class_probability}) are intractable. Thus, we have to use various techniques to approximate these integrals. 

For example, one can use laplace approximation method, which builds a Gaussian approximation $q(f | X, y)$ to the true posterior $p(f | X, y)$. This approximation is obtained, by performing the Taylor expansion of the function $\log p(f | X, y)$ around it's maximum $\hat f$. 

Substituting this Gaussian approximation back into (\ref{classification_conditional}) and (\ref{classification_class_probability}), we obtain tractable integrals, and can compute the predictive distribution in a closed form. The more detailed derivation of this algorithm and another algorithm, based on Expectation Propagation can be found in \cite{GPinML}.

We will also describe another method for GP-classification below.

Computational complexity of computing the predictive distribution for this method scales as $\bigO(n^3)$.
	
\subsection{Kernel functions}
	To be wrritten.
	
\subsection{Hyper-parameter estimation}
	Bayesian paradigm provides a way of estimating the kernel hyper-parameters of the GP-model through maximizization of the marginal likelihood of the model. Marginal likelihood is given by
$$p(y | X) = \int p(y | f, X) p(f | X) df,$$
which is the likelihood, marginalized over the hidden values $f$ of the underlying process.

For the GP-regression the marginal likelihood can be computed in claused form and is given by
\begin{equation}
	\label{regression_ml}
	\log p(y | X) = -\frac 1 2 y^{T} (K + \sigma_n^2 I)^{-1} y - \frac 1 2 \log |K + \sigma_n^2 I| - \frac n 2 \log 2 \pi.
\end{equation}

For the method, described in section \ref{gp-classification} the marginal likelihood can be computed as follows.
$$p(y | X) = \int p(y | f, X) p(f | X) df = \int \exp{\Psi(f)} df,$$
where we use the notation from section \ref{gp-classification}. Using the Taylor expansion, locally near $\hat f$ we have $\Psi(f) \simeq \Psi(\hat f) + \frac 1 2(f - \hat f)^T A (f - \hat f)$, where $A$ is the hessian of $\Psi$ at $\hat f$. Using this approximation we obtain
$$p(y | X) \simeq q(y | X) = \exp(\Psi(\hat f)) \int \exp( - \frac 1 2 (f - \hat f)^T A (f - \hat f)) df.$$
This last integral can be calculated analytically to obtain a closed form approximation to the log marginal likelihood. 
\begin{equation}
	\label{classification_ml}
	\log q(y|X) = -\frac 1 2 \hat f^T K^{-1} \hat f + \log p(y|\hat f) - \frac 1 2 \log|B|,
\end{equation}
where 
$$|B| = |K| \left|- \left. \frac{\partial^2 \log p(y | f)}{\partial f^2} \right|_{f = \hat f} \right|.$$ 

Using the derived formulas \ref{classification_ml} and \ref{regression_ml} we can find the optimal values of hyper-parameters through maximization of the marginal likelihood of the corresponding model.
	
\subsection{Theoretical perspectives}
	\hspace{0.6cm}To be wrritten.

\pagebreak
\section{Review of existing methods}
	It follows from the discussion above, that full Gaussian process regression scales as $O(n^3)$ and thus cannot be applied to big datasets. In this section we will review several approximate methods, that make Gaussian processes practical.

\subsection{Methods, based on inducing inputs}
	Most of the existing methods are based on introducing a set of $m$ function points that are called inducing inputs. Using these inputs one can make approximate predictions of the values of the hidden process at test points with a complexity of $O(nm^3)$ instead of $O(n^3)$.
	
	Consider the following situation. We have a dataset of $n$ examples $x_i$ with corresponding values $y_i$. We will denote the matrix of pairwise values of the covariance function by $K_{nn}$. Now we introduce a set of $m$ inducing inputs. We will denote the corresponding covariance matrix by $K_{mm}$ and the matrices of covariances between the inducing points and training points by $K_{nm}$ and $K_{mn}$. We will denote the vectors, comprised of noisy and true function values $y_i$ and $f_i$ at training points by $y$ and $f$ respectively. We will also introduce a distribution $q(u)$ over the hidden function values $u$ at the inducing inputs.
	
	It's easy to see, that
	$$p(y|f) = \N (y|f, \sigma_n I),$$
	$$p(f|u) = \N (f|K_{nm} K_{mm}^{-1}u, \tilde K),$$
	$$p(u) = \N(u|0, K_{mm}),$$
	where $\tilde K = K_{nn} - K_{nm} K_{mm}^{-1} K_{mn}.$
		
	\subsubsection{Variational learning of inducing points}
		\label{Titsias}
		
		The method discussed here was introduced in \cite{Titsias}. This method provides a way to find the optimal positions of the inducing points, as well as the optimal distribution of the process value at these points.

		Let $z$ denote a vector comprized of the process values at some new points. We can calculate the predictive distribution at these points as follows
		$$p(z|y) = \int p(z|f) p(f|y) df.$$
		Let's fix the inducing point positions $x_1, \ldots, x_m$. As above, $u$ is the vector compised of the process values at these points. We can rewrite the above equation
		\begin{equation}
			\label{predictive1}
			p(z|y) = \iint p(z|u, f) p(f| u, y) p(u|y)df du,
		\end{equation}
		% $$p(z | y) = \iint p(z | u, f) p(f | u, y)df du,$$
		as $p(z|u, f, y) = p(z|u, f)$. 

		Suppose for a moment, that $u$ is a sufficient statistics for the parameter $f$ in the scence that $z$ and $f$ are conditionally independent given $u$. Then we have 
		$$p(z|f, u) = \frac {p(z, f|u)} {p(f|u)} = \frac {p(z | u) p(f | u)}{p(f|u)} = p(z|u),$$
		$$p(z|y, u) = \frac {p(z, y, u)}{p(y, u)} = \frac {\int p(y|f)p(f, z, u) du}{\iint p(y|f) p(f, z, u) df dz} = \frac {\int p(y|f) p(z|u) p(u|f) p(f)df}{\iint p(y|f) p(z|u) p(u|f) p(f)df dz} = $$
		$$= \frac {\int p(y|f)p(f)p(u|f)df \cdot p(z|u)} {\int p(y|f)p(f)p(u|f)df \cdot \int p(z|u) dz} = \frac{\int p(y, f) p(u|f) df} {\int p(y, f) p(u|f) df} p(z|u) = p(z|u).$$

		So, $p(z|y, u) = p(z|u)$. If we set the points, corrwsponding to the process values $z$, to the traing points, we will have $z = f$, and thus $p(f|y, u) = p(f|u)$.

		Substituting these formulas into (\ref{predictive1}) we achieve
		$$q(z) = p(z|y) = \iint p(z|u) p(f|u) p(u|y)df du = \iint p(z|u) p(u|y) du = $$
		\begin{equation}
			\label{predictive2}
			= \int p(z|u)\varphi(u) du  = \int q(z, u) du, 
		\end{equation}
		where $\varphi(u) = p(u|y)$, $q(z, u) = p(z|u)\varphi(u)$.

		In practice however it's difficult to guarantee that $u$ is a sufficient statistics. Thus we can only expect $q(z)$ to be an approximation to $p(z|y)$. In such case we can choose $\varphi(u)$ to be a variational distribution, where in general $\varphi(u) \ne p(u | y)$. We will consider $\varphi(u)$ to be Gaussian with a mean vector $\mu$ and covariance matrix $\Sigma$.

		By using the eq. (\ref{predictive2}) we can calculate the approximate posterior GP mean at point $x$ and covariance at points $x, x'$
		$$\E[z(x)] = K_{xm} K_{mm}^{-1} \mu,$$ 
		$$\cov(z(x), z(x')) = k(x, x') - K_{xm} K_{mm}^{-1} K_{mx'} + K_{xm} A K_{mx'},$$
		where $A = K_{mm}^{-1} \Sigma K_{mm}^{-1}$.

		Now we have to specify a way to find the variational distribution parameters $\mu$ and $\Sigma$, and the inducing input positions $X_m$ and a way to optimize the kernel hyper-parameters. 
		% In order to do so, we will form the variational distribution $q(f)$ and the exact posterior $p(f|y)$ on the training function values, and then minimize the distance between this two distributions. Equivalently, we can minimize a distance, between the augmented true posterior $p(f, u|y)$ and $q(f, u)$.
		In order to do so, we will form the variational distribution $q(f, u)$ and the exact posterior $p(f, u|y)$ on the training function values and the values at the inducing points, and then minimize the KL-divergence between these two distributions. This minimization is equivalently expressed as the maximization of the following lower bound of the true marginal likelihood:
		$$F_V(X_m, \varphi) = \iint p(f|u) \varphi(u) \log \frac{p(y|f) p(u)}{\varphi(u)} df du.$$
		This bound can be optimized analytically with respect to $\phi$. The optimal distribution $\varphi(u) \sim \N(u|\hat u, \Lambda^{-1})$, where
		$$\Lambda = \frac 1 {\sigma_n} K_{mm}^{-1} K_{mn} K_{nm} K_{mm}^{-1} + K_{mm}^{-1},$$
		$$\hat u = \frac 1 {\sigma_n} \Lambda^{-1} K_{mm}^{-1} K_{mn} y.$$
		Substituting the optimal values of variational parameters into the $F_V$ we obtain the following bound
		$$F_V(X_m) = \log \N(y|0, \sigma_n^2 I + K_{nm} K_{mm}^{-1} K_{mn}) - \frac 1 {2\sigma_n^2} \tr(\tilde K).$$

		Another derivation of this lower bound is provided in section (\ref{svi}).

		The bound $F_V(X_m)$ is computed in $o(nm^2)$ time. Now we will calculate it's gradient in order to be able to maximize it with respect to $X_m$ and kernel hyper-parameters. We will denote $B = \sigma_n^2 I + K_{nm} K_{mm}^{-1} K_{mn}$. Then
		$$F_V(X_m, \theta, \sigma_n) = -\frac 1 2 \left(n \log 2\pi + \log |B| + y^T B^{-1} y + \frac 1 {\sigma_n^2} \tr(\tilde K)\right),$$
		$$\derivative{F_V}{\theta} = \frac 1 2 \left( -\tr \left(B^{-1} \derivative{B}{\theta}\right) + y^T B^{-1} \derivative{B}{\theta} B^{-1} y - \right.$$    
		$$- \left. \frac 1 {\sigma_n^2} \tr\left(\derivative{K_{nn}}{\theta} - \left(\derivative{K_{nm}}{\theta}K_{mm}^{-1} - K_{nm} K_{mm}^{-1} \derivative{K_{mm}}{\theta}K_{mm}^{-1}\right) K_{mn} - K_{nm} K_{mm}^{-1} \derivative{K_{mn}}{\theta}\right)\right),$$
		where
		$$\derivative{B}{\theta} = \left(\derivative{K_{nm}}{\theta}K_{mm}^{-1} - K_{nm} K_{mm}^{-1} \derivative{K_{mm}}{\theta}K_{mm}^{-1}\right) K_{mn} +  K_{nm} K_{mm}^{-1} \derivative{K_{mn}}{\theta}.$$

		We can rewrite
		$$\derivative{F_V}{\theta} = \frac 1 2 \left( -\tr \left(B^{-1} \derivative{B}{\theta}\right) + y^T B^{-1} \derivative{B}{\theta} B^{-1} y - \frac 1 {\sigma_n^2} \tr \left(\derivative {K_{nn}} {\theta} - \derivative {B}{\theta}\right) \right).$$

		Now we can optimize $F_V$ with respect to kernel hyper-parameters. Similarly, we can take derivatives with respect to $X_m$ and $\sigma_n$ and opptimize $F_V$ with respect to them as well.

		However, if we compute $F_v$ and it's derivatives as they are, it takes $O(n^3)$ time which is not faster, than recovering the full Gaussian process. So, we have to rewrite these values in a form that allows for faster computation.

		First of all, let's deal with $\log|B|$ and $B^{-1}$. Using the matrix determinant lemma we obtain
		$$|B| = |\sigma_n^2 I + K_{nm} K_{mm}^{-1} K_{mn}| = \frac{\left|K_{mm} + \cfrac{K_{mn} K_{nm}}{\sigma_n^2}\right| \sigma_n^2}{|K_{mm}|}.$$
		So, denoting $A = K_{mm} + \cfrac{K_{mn} K_{nm}}{\sigma_n^2}$, we obtain
		$$\log |B| = \log |A| + 2 \log \sigma_n - \log |K_{mm}|.$$
		Note tha this is computed in $O(n m^2)$ instead of $O(n^3)$.

		Using the Woodbury identity, we obtain
		$$B^{-1} = (\sigma_n^2 I + K_{nm} K_{mm}^{-1} K_{mn})^{-1} = \frac I {\sigma_n^2} - \frac{K_{nm} A^{-1} K_{mn}}{\sigma^{4}},$$
		which allows for computing $y^T B^{-1} y$ in $O(n m)$.

		Similarly, we can compute the gradient in $O(nm^2)$. In order to do so, we need to rewrite every trace $\tr(M_{nm} M_{mm} M_{mn})$, where $M_{kl} \in \R^{k \times l}$, in the form $\tr(M_{mm} M_{mn} M_{nm})$, which is computed in $O(nm^2)$, and use the derived formulas for $B^{-1}$.

		Now let's try to compute the second order derivatives.
		$$\frac{\partial^2 F_V} {\partial \theta_j \partial\theta_i} = \derivative{}{\theta_j} \left(\derivative{F_V}{\theta_i}\right) = \frac 1 2 \derivative{}{\theta_j} \left( -\tr \left(B^{-1} \derivative{B}{\theta_i}\right) + y^T B^{-1} \derivative{B}{\theta_i} B^{-1} y - \frac 1 {\sigma_n^2} \tr \left(\derivative {K_{nn}} {\theta_i} - \derivative {B}{\theta_i}\right)\right) = $$
		$$ = \frac 1 2 \left(\tr\left( B^{-1} \derivative{B}{\theta_j} B^{-1}\derivative{B}{\theta_i} - B^{-1} \sndderivative{B}{\theta_j}{\theta_i}\right) + y^T \left(B^{-1} \sndderivative{B}{\theta_j} {\theta_i} B^{-1}  - 2 B^{-1} \derivative{B}{\theta_j} B^{-1} \derivative{B}{\theta_i} B^{-1} \right) y \right. - $$
		$$ \left.- \frac 1 {\sigma_n^2} \tr\left(\sndderivative{K_{nn}}{\theta_j}{\theta_i} - \sndderivative{B}{\theta_j}{\theta_i}\right)\right),$$
		where
		$$\sndderivative{B}{\theta_j}{\theta_i} = \derivative{}{\theta_j} \left(\derivative{K_{nm}}{\theta_i}K_{mm}^{-1} K_{mn} - K_{nm} K_{mm}^{-1} \derivative{K_{mm}}{\theta_i}K_{mm}^{-1} K_{mn} +  K_{nm} K_{mm}^{-1} \derivative{K_{mn}}{\theta_i}\right) = $$

		$$ = \sndderivative{K_{nm}}{\theta_j}{\theta_i}K_{mm}^{-1} K_{mn} + K_{nm} K_{mm}^{-1}\sndderivative{K_{mn}}{\theta_j}{\theta_i}  - \derivative{K_{nm}}{\theta_i}K_{mm}^{-1} \derivative{K_{mm}}{\theta_j}K_{mm}^{-1}K_{mn} - $$
		
		$$- K_{nm} K_{mm}^{-1} \derivative{K_{mm}}{\theta_j}K_{mm}^{-1}\derivative{K_{mn}}{\theta_i} + \derivative{K_{nm}}{\theta_j}K_{mm}^{-1} \derivative{K_{mn}}{\theta_i} + \derivative{K_{nm}}{\theta_i} K_{mm}^{-1} \derivative{K_{mn}}{\theta_j} $$

		$$- \derivative{K_{nm}}{\theta_j} K_{mm}^{-1} \derivative{K_{mm}}{\theta_i}K_{mm}^{-1} K_{mn} + K_{nm} K_{mm}^{-1}\derivative{K_{mm}}{\theta_j} K_{mm}^{-1}\derivative{K_{mm}}{\theta_i}K_{mm}^{-1} K_{mn}$$

		$$ - K_{nm} K_{mm}^{-1} \sndderivative{K_{mm}}{\theta_j}{\theta_i}K_{mm}^{-1} K_{mn} + K_{nm} K_{mm}^{-1} \derivative{K_{mm}}{\theta_i}K_{mm}^{-1}\derivative{K_{mm}}{\theta_j}K_{mm}^{-1} K_{mn} - $$

		$$- K_{nm} K_{mm}^{-1} \derivative{K_{mm}}{\theta_i}K_{mm}^{-1} \derivative{K_{mn}}{\theta_j}.$$


	\pagebreak
	\subsubsection{Stochastic variational inference}
		In this subsection we describe a method for maximizing the lower bound (\ref{main_elbo}) in case of the GP-regression problem, which was proposed in \cite{BigData}. While the method described in the previous section is much faster then the full GP-regression, it's complexity is still rather big. We could try to reduce the time consumption of optimizing the lower bound by using stochastic optimization methods. However, the function in the right-hand side of (\ref{titsias_elbo}) does not have a form of sum over objects, and thus it's not clear, how to apply the stochastic methods.

However, the original bound from (\ref{main_elbo}) does have a form of sum over objects, and we can thus apply stochastic methods to it. In the regression case the expectations in the bound (\ref{main_elbo}) are tractable. In this case, we can rewrite the bound as

$$\log p(y) \ge \sum_{i = 1}^{n} \left( \log \N(y_i | k_i^T K_{mm}^{-1} \mu, \nu^2) - \frac 1 {2 \nu^2} \tilde K_{ii} - \frac 1 2 \tr (\Sigma \Lambda_i) \right) - $$

\begin{equation} \label{svi_elbo}
	-\frac 1 2 \left (\log \frac {|K_{mm}|} {|\Sigma|} - m + \tr(K_{mm}^{-1} \Sigma) + \mu^T K_{mm}^{-1} \mu \right),
\end{equation}
where $\Lambda_i = \frac 1 {\nu^2} K_{mm}^{-1} k_i k_i^T K_{mm}^{-1}$, and $k_i = K(x_i, Z)$ is the vector of covariances between the $i$-th data point and inducing points.

In the \lstinline{svi} method we directly optimize this ELBO with respect to both variational parameters and kernel hyper-parameters in a stochastic way. The authors of the method suggest to use the stochastic gradient descent with natural gradients for the variational parameters and usual gradients for kernel hyper-parameters.

Natural gradients are gradients with respect to the natural parameters of an exponential family of distributions. These gradients are considered to be effective in case of optimization with respect to probability distribution parameters, because they use symmetrized KL divergence between the distributions instead of usual distance between distribution parameters as a distance metric. For more information about natural gradients see, for example \cite{ExpFamilyGeom}.

The complexity of computing a stochastic update of variational and kernel parameters is independent of $n$ and scales as $\bigO(m^3)$. The complexity of one pass over data (epoch) is thus $\bigO(nm^3)$ which is worse, than the corresponding complexity of the \lstinline{vi} method. However, the stochastic optimization might give this method advantage against the \lstinline{vi} method, because stochastic optimization some times leads to faster convergence in big data problems. 

However, for big problems the number of required inducing points $m$ is usually quite big. The number of parameters we have to optimize scales as $\bigO(m^2)$. Indeed, we need to optimize the bound with respect to the variational parameters $\mu$ ($m$ parameters) and $\Sigma$ ($\frac {m(m + 1)} 2$ parameters), and with respect to kernel hyper-parameters. This makes the optimization problem of the \lstinline{svi} method much harder than the one we have to solve in the \lstinline{vi} method (where we only have to optimize the bound with respect to kernel hyper-parameters). We will compare the two methods in the experiments section.


	\subsection{Stochastic variational inference for classification}
		\label{svi_classification}
Finally, we can apply the bound (\ref{main_elbo}) to the classification problem. In this case, we can't analytically compute the expectations $\E_{q(f_i)} \log p(y_i | f_i)$. However, this expectations are one-dimensional Gaussian integrals and can thus be approximated with a wide range of techniques. 

The method was proposed in \cite{SVIclassification}. The authors suggest to use Gauss-Hermite quadratures in order to approximate the expectations in (\ref{main_elbo}) and their derivatives. 

For this method stochastic optimization is applicable, which makes it suitable for big data problems.

% For small and moderate problems one can use \lstinline{L-BFGS-B} optimization method in order to maximize the bound with respect to variational parameters and kernel hyper-parameters. For big problems stochastic optimization can be applied.

	\subsection{Variational inference for classification}
		\subsubsection{Jaakkola-Jordan lower bound}
			We've described two approaches to optimizing the lower bound (\ref{explicit_svi_elbo}) in case of the regression problem. The optimization problem, that we have to solve in the \lstinline{svi} method seems to be much harder, than the one, that we have to solve in the \lstinline{vi} method, although we can solve the former with stochastic optimization techniques. In this subsection we will devise an approach, analogues to the \lstinline{vi-means} method for the classification problem.

The problem of optimizing the lower bound (\ref{explicit_svi_elbo}) with respect to the variational parameters $\mu$ and $\Sigma$ is very similar to the Bayesian logistic regression problem with Gaussian prior over the parameters. In \cite{JaakkolaJordan} a method, that implies a closed form approximation to the posterior distribution over the parameters. Applying this method, we can avoid optimization with respect to the variational parameters and use analytical formulas, similar to the ones, used in the \lstinline{vi-means} method.

Article \cite{JaakkolaJordan} provides the following lower bound for the logarithm of logistic function/.
$$\log g(x) = - \log(1 + \exp(-x)) \ge \frac x 2 - \frac \xi 2 + \log g(\xi) - \frac 1 {4 \xi} \tanh\left(\frac \xi 2 \right)(x^2 - \xi^2).$$
This bound becomes tight, when $\xi = x$.
We will denote $$\lambda(\xi) = \frac {\tanh\left(\frac\xi 2\right)}{4 \xi}.$$
This implies
$$\log g(x) \ge \frac x 2 - \frac \xi 2 + \log g(\xi) - \lambda(\xi) (x^2 - \xi^2)$$

Substituting this bound back to (\ref{explicit_svi_elbo}) we obtain
$$\log p(y) \ge \sum_{i = 1}^{n} \E_{q(f_i)} \log p(y_i | f_i) - \KL{q(u)} {p(u)} = \sum_{i = 1}^{n} \E_{q(f_i)} \log g(y_i f_i) - \KL{q(u)} {p(u)} \ge $$
$$\ge \sum_{i = 1}^{n}\left(\E_{q(f_i)} \left [\log g(\xi_i) + \frac {y_i f_i - \xi_i} {2} - \lambda(\xi_i) (f_i^2 - \xi_i^2) \right]\right) - \KL{q(u)} {p(u)} = $$
$$= \sum_{i = 1}^{n} \left(\log g(\xi_i) + \frac {y_i m_i - \xi_i} {2}  + \lambda(\xi_i) \xi_i^2 - \lambda(\xi_i) (m_i^2 + S_i^2) \right) - \KL{q(u)} {p(u)} = $$
$$= \sum_{i = 1}^{n} \left(g(\xi_i) - \frac {\xi_i}{2} + \lambda(\xi_i) \xi_i^2\right) + \frac 1 2 \mu^T K_{mm}^{-1} K_{mn} y - \tr\left(\Lambda(\xi) (K_{nn} + K_{nm} K_{mm}^{-1} (\Sigma - K_{mm}) K_{mm}^{-1} K_{mn})\right) -$$
$$- \mu^T K_{mm}^{-1} K_{mn} \Lambda(\xi) K_{nm} K_{mm}^{-1} \mu - \KL{q(u)} {p(u)} = J(\mu, \Sigma, \xi, \theta),$$
where 
$$\Lambda(\xi) = 
\left(
\begin{array}{cccc}
	\lambda(\xi_1) & 0 & \ldots & 0 \\
	0 & \lambda(\xi_2) & \ldots & 0 \\
	\ldots & \ldots & \ldots & \ldots \\
	0 & 0 & \ldots & \lambda(\xi_n) \\
\end{array}
\right).
$$

Differentiating $J$ with respect to $\mu$ and $\Sigma$ and setting the derivatives to zero, we obtain
\begin{equation}\label{vi_optimal_sigma}
	\hat \Sigma(\xi) = (2 K_{mm}^{-1} K_{mn} \Lambda(\xi) K_{nm} K_{mm}^{-1} + K_{mm}^{-1})^{-1},
\end{equation}
\begin{equation}\label{vi_optimal_mu}
	\hat \mu(\xi) = \frac 1 2 \hat \Sigma(\xi) K_{mm}^{-1} K_{mn} y.
\end{equation}
Note, that these formulas are very similar to the corresponding optimal values in the regression problem.

We now apply coordinate-wise optimization to tune both $\mu$, $\Sigma$ and $\xi$. On the first step we use formulas (\ref{vi_optimal_sigma}) and (\ref{vi_optimal_mu}) to find the optimal distribution over $f$ for the current values $\xi_{old}$ of $\xi$. On the second step we maximize $J$ with respect to $\xi$ for fixed $\mu$ and $\Sigma$. This leads to
$$\xi_i^2 = \E_{q(f | \xi_{old})} f_i^2.$$
Now, performing a few updates of $\mu$, $\Sigma$ and $\xi$, we obtain closed-form formulas for optimal
$\mu$ and $\Sigma$ and can substitute them back to the ELBO.

		\subsubsection{Taylor decomposition approximation}
			\input{vi_taylor_classification.tex}
	\pagebreak

\section{Experiments}
	In this section we compare the methods, described above, for both regression and classification problems. We will compare the methods, that use inducing points with the standard methods and also compare the different versions of \lstinline{vi} and \lstinline{svi} methods to each other.

All of the plots, apart from the plots in the next subsection, have title of the following format.
$$\mbox{[name of the dataset]}, n = \mbox{[number of objects in the training set]},$$
$$d = \mbox{[number of features]}, m = \mbox{[number of inducing inputs]}$$

The plots in the next subsection do not have the $m$ in the title, because this parameter is not fixed in the corresponding experiments. If the name of the dataset is ``generated'', it means that the dataset was sampled from some Gaussian process.

For the regression problem we use the $R^2$ score, which is given by
$$R^2(y, \hat y) = 1 - \frac{\sum_{i = 1}^{n} (y_i - \hat y_i)^2}{\sum_{i = 1}^{n} (y_i - \bar y_i)^2},$$
where $\bar y_i = \frac 1 n \sum_{i = 1}^n y_i$. Here $y$ is the vector of true answers on the test set, and $\hat y$ is the vector of predicted answers.

For the classification problem we use accuracy score.

In all the experiments we used the squared exponential covariance function. 

\subsection{Inducing input methods and standard methods}
	% \begin{figure}[!t]
% 	\centering
% 	\subfloat{
% 		\scalebox{0.8}{
% 			\input{../../Code/Experiments/plots/inducing_inputs/d1_n500.pgf}
% 		}
% 	}
% 	\subfloat{
% 		\scalebox{0.8}{
% 			\input{../../Code/Experiments/plots/inducing_inputs/d5_n500.pgf}
% 		}
% 	}

% 	\subfloat{
% 		\scalebox{0.8}{
% 			\input{../../Code/Experiments/plots/inducing_inputs/d10_n4000.pgf}
% 		}
% 	}
% 	\subfloat{
% 		\scalebox{0.8}{
% 			\input{../../Code/Experiments/plots/inducing_inputs/abalone.pgf}
% 		}
% 	}
% 	\caption{The dependence between prediction quality and the number of inducing inputs for the regression problem}
% 	\label{ind_points_results}
% \end{figure}

\begin{figure}[!t]
	\centering
	\subfloat{
		\scalebox{0.8}{
			\input{../../Code/Experiments/plots/inducing_inputs/d5_n500.pgf}
		}
	}
	\subfloat{
		\scalebox{0.8}{
			\input{../../Code/Experiments/plots/inducing_inputs/class_d2_n200.pgf}
		}
	}
	\caption{The dependence between the prediction quality and the number of inducing inputs for standard and inducing point methods}
	\label{ind_vs_std}
\end{figure}


We've seen above, that inducing input methods have a much smaller computational complexity than the standard methods for both GP-regression and GP-classification problems. In this section we empirically compare the prediction quality on the test data for these methods with the quality obtained by the standard methods.

We also explore the dependence between the number of inducing points used by the method and the prediction quality.

For the regression problem we use two variations of the \lstinline{vi} method. \lstinline{vi-means} method does not maximize the lower bound with respect to the positions of inducing inputs and just uses the K-Means cluster centers as the positions of inducing inputs. The \lstinline{vi} method on the other hand does optimize for the inducing input positions. \lstinline{full GP} method is the standard GP-regression method, described in section \ref{gp_regression}.

Fig. \ref{ind_vs_std} shows the dependence between the prediction quality and the number of inducing inputs for \lstinline{vi} methods for regression and classification problems.

As we can see, optimization with respect to the positions of inducing inputs does not dramatically increase the quality of predictions in the provided experiment. It does however make the optimization problem that we have to solve much harder. In general, the optimization of inducing input positions does increase the prediction quality, but makes the method much slower. We will thus abandon this method and only use \lstinline{vi-means} in other experiments.

% We can also see from these plots that for sufficient number of inducing inputs the \lstinline{vi} methods reach the predictive quality of the standard methods. 

% In the bottom plots of fig. \ref{ind_points_results}, the dependence between the number of used inducing inputs and the quality of the \lstinline{vi-means} method predictions is shown for two bigger datasets. 

% \begin{figure}[!t]
% 	\centering
% 	\subfloat{
% 		\scalebox{0.8}{
% 			\input{../../Code/Experiments/plots/inducing_inputs/class_d2_n200.pgf}
% 		}
% 	}
% 	\subfloat{
% 		\scalebox{0.8}{
% 			\input{../../Code/Experiments/plots/inducing_inputs/class_german.pgf}
% 		}
% 	}
% 	\caption{The dependence between prediction quality and the number of inducinge inputs for the classification problem}
% 	\label{ind_inputs_class_results}
% \end{figure}

\begin{figure}[!t]
	\centering
	\subfloat{
		\scalebox{0.8}{
			\input{../../Code/Experiments/plots/inducing_inputs/abalone.pgf}
		}
	}
	\subfloat{
		\scalebox{0.8}{
			\input{../../Code/Experiments/plots/inducing_inputs/class_german.pgf}
		}
	}
	\caption{The dependence between prediction quality and the number of inducing inputs for \lstinline{vi-means} and \lstinline{svi-classification} methods}
	\label{ind_inputs_results}
\end{figure}

Fig. \ref{ind_inputs_results} shows the dependence between the prediction quality and the number of inducing inputs for lstinline{vi-means} and \lstinline{svi-classification} methods. As we can see, the prediction quality gets better as the number of inducing inputs grows. However, it's hard to say, how many inducing points one should use in practice. The best answer is probably the biggest amount one can afford to train. 

% In fig. \ref{ind_inputs_class_results} the results of similar experiments for the classification problem are provided. Here we compare the \lstinline{Laplace} method, which was described in section \ref{gp-classification}, and the \lstinline{svi-classification} method, which was described in section \ref{svi_classification}.

% As we can see, on the chosen dataset the \lstinline{svi-classification} method can not reach the predictive quality of the \lstinline{Laplace} method even for reasonably big values of $m$. However, the time consumption of the \lstinline{Laplace} method does not allow to use it even for moderate~problems. 




\subsection{SVI method variations}
	\begin{figure}[t!]
	\centering

	\subfloat{
		\scalebox{0.75}{
			\input{../../Code/Experiments/Plots/svi_variations/small_generated.pgf}
		}
	}
	\subfloat{
		\scalebox{0.75}{
    		\input{../../Code/Experiments/Plots/svi_variations/small_real.pgf}
		}
	}
	\vspace{0.1cm}
	\subfloat{
		\scalebox{0.75}{
			\input{../../Code/Experiments/Plots/svi_variations/medium_generated.pgf}
		}
	}
	\subfloat{
		\scalebox{0.75}{
    		\input{../../Code/Experiments/Plots/svi_variations/medium_real.pgf}
		}
	}
	\caption{\lstinline{svi} methods' performance on small and medium datasets}
	\label{svi_results}
\end{figure}
In this section we compare several variations of the \lstinline{svi} method for the regression problem.

The first variation is denoted by \lstinline{svi-natural}. It is the method as it was proposed in \cite{BigData}. It uses stochastic gradient descent with natural gradients for minimizing the ELBO with respect to the variational parameters, and usual gradients with respect to kernel hyper-parameters.

The methods \lstinline{svi-L-BFGS-B} and \lstinline{svi-FG} use the same lower bound (\ref{svi_elbo}) and optimize it with deterministic optimization methods L-BFGS-B and projected gradient respectively. We use the bound-constrained optimization methods, because the hyper-parameters of the squared exponential kernel must be positive.

We can not use the natural gradients in this setting, because they are not necessarily a descent direction and can't be used by L-BFGS-B or gradient descent. Thus, we use usual gradients with respect to variational parameters $\mu$ and $\Sigma$ for these methods. However, the matrix $\Sigma$ has to be symmetric and positive definite and we have to ensure that our optimization updates maintain these properties. In order to avoid complex constrained optimization problems, we use Cholesky decomposition of $\Sigma$ and optimize the bound with respect to the Cholesky factor $L_\Sigma$ of $\Sigma$. This allows us to solve a simpler bound-constrained problem instead of a general constrained optimization problem.

Finally, the \lstinline{svi-SAG} uses stochastic average gradient method to minimize the ELBO. This method also uses Cholesky factorization and usual gradients instead of natural for the same reasons. For more information about SAG method see \cite{SAG}.


The results on small and medium datasets are shown in fig. \ref{svi_results}.

As we can see, on these moderate problems using stochastic optimization does not give any advantages against the L-BFGS-B method. However, using the natural gradients allows the stochastic gradient descent method to beat SAG. For these reasons we will only use the \lstinline{svi-natural} and \lstinline{svi-L-BFGS-B} methods in the comparison with the \lstinline{vi-means} method.
\subsection{VI method variations}
	\begin{figure}[t!]
	\centering
	\subfloat{
		\scalebox{0.75}{
			\input{../../Code/Experiments/Plots/vi_variations/small_real.pgf}
		}
	}
	\subfloat{
		\scalebox{0.75}{
    		\input{../../Code/Experiments/Plots/vi_variations/medium_real.pgf}
		}
	}

	% \subfloat{
	% 	\scalebox{0.75}{
	% 		\input{../../Code/Experiments/Plots/vi_variations/big_real.pgf}
	% 	}
	% }
	% \subfloat{
	% 	\scalebox{0.75}{
	% 		\input{../../Code/Experiments/Plots/vi_variations/huge_real.pgf}
	% 	}
	% }
	\caption{ \lstinline{vi} method variations on different datasets}
	\label{vi_results}
\end{figure}

In this section we compare two optimization methods for the \lstinline{vi-means} method.

The first variation is denoted by \lstinline{Projected Newton}. It uses projected Newton method for minimizing the ELBO (\ref{titsias_elbo}). The second variation is denoted by \lstinline{means-L-BFGS-B} and uses L-BFGS-B optimization method.

\lstinline{Projected Newton} method uses finite-difference approximation of the hessian. It also makes hessian-correction in order to make it symmetric and positive-definite. The optimization method itself makes a Newton step and then projects the result to the feasible set in the metric, determined by the hessian. For more information about the method see for example \cite{ProjNewton}.

The time complexity of one iteration for both projected Newton method and L-BFGS-B is $\bigO(nm^2)$. In the projected Newton method we have to compute the hessian matrix of the ELBO with respect to covariance hyper-parameters. In case of squared exponential covariance function the time, needed to compute the hessian, is twice the time, needed to compute the gradient.

The results are provided in fig. \ref{vi_results}. In the provided experiments projected Newton method beats L-BFGS-B. However, the results are close and on different datasets L-BFGS-B beats projected Newton. We need to perform further experiments in order to find out whether using second order optimization provides benefits in the \lstinline{vi} method.

In further experiments we use the L-BFGS-B method because it was more stable in general in our experiments.
\subsection{Comparison of VI and SVI methods}
	In this section we compare the \lstinline{vi-means} method (with L-BFGS-B optimization method) with \lstinline{svi-L-BFGS-B} and \lstinline{svi-natural}. We've described these methods above. 

We've seen, that the computational complexity of one epoch of the \lstinline{vi-means} method is better method than it is for the \lstinline{svi-natural} and \lstinline{svi-L-BFGS-B}. However, the \lstinline{svi-natural} method uses stochastic optimization, which might lead to faster convergence in big data problems.

\begin{figure}[!h]
	\centering
	\subfloat{
		\scalebox{0.75}{
	    	\input{../../Code/Experiments/Plots/vi_vs_svi/medium_real.pgf}
		}
	}
	\subfloat{
		\scalebox{0.73}{
			\input{../../Code/Experiments/Plots/vi_vs_svi/1e5_sg_lbfgs.pgf}
		}
	}
	\caption{\lstinline{vi} and \lstinline{svi} methods comparison}
	\label{visvi_results}
\end{figure}

Fig. \ref{visvi_results} shows the results of the experimental comparison of \lstinline{vi} and \lstinline{svi} methods. In the first experiment of these two we didn't run the \lstinline{svi-natural}, because we've already compared it to the \lstinline{svi-L-BFGS-B} method on this exact dataset and it proved to be worse (see fig. \ref{svi_results}).

We can see, that \lstinline{svi-natural} performs slightly better, then the deterministic \lstinline{svi-L-BFGS-B}, but can't beat \lstinline{vi-means}. The reason for that is that the optimization problem for the \lstinline{svi} method is much harder then the one, solved by the \lstinline{vi} method. Indeed, for $m = 1000$ and squared exponential kernel we have about $5 \cdot 10^5$ optimization parameters for \lstinline{svi} methods and just $3$ parameters for the \lstinline{vi} method.


\pagebreak
\begin{thebibliography}{99}

	\bibitem{Titsias}
	Titsias M. K. (2009).  Variational Learning of Inducing Variables in Sparse Gaussian
	Processes.  In: {\it International Conference on Artificial Intelligence and Statistics}, pp.~567–574.

	\bibitem{BigData}
	Hensman J., Fusi N., Lawrence D. (2013).  Gaussian Processes for Big Data.  In: {\it Proceedings of the Twenty-Ninth Conference on Uncertainty in Artificial Intelligence}.

	\bibitem{SVIclassification}
	Hensman J., Matthews G., Ghahramani Z. (2015). Scalable Variational Gaussian Process Classification.  In: {\it Proceedings of the Eighteenth International Conference on Artificial Intelligence and Statistics}.

	\bibitem{JaakkolaJordan}
	Jaakkola T., Jordan M. (1996). A Variational Approach to Bayesian Logistic Regression Models and Their Extensions. In: {\it Artificial Intelligence and Statistics}.

\end{thebibliography}	
\end{document}