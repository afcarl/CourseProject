\documentclass[12pt]{article}

\usepackage[utf8]{inputenc}
\usepackage[russian]{babel}
\usepackage[T2A]{fontenc}
\usepackage{textcomp}
\usepackage{a4wide}
\usepackage{amsmath, amssymb}
\usepackage{graphicx}
\usepackage{wrapfig}
\usepackage{caption}
\usepackage{subfig}
\usepackage{listings}
\usepackage{hyperref}
% \usepackage{fontspec}
\usepackage{pgfplots}
\usepackage{tikz}
\usepackage{amsthm}
\usepackage{pgf,pgfarrows,pgfnodes}
\usepackage{pgf}

\lstset{
language=Python,
basicstyle=\ttfamily\small,
otherkeywords={self},                   
}

\title{Title}
\title{Неточный метод Ньютона.}
\date{4 октября 2015}
\author{Павел Измаилов}

\begin{document}

\renewcommand{\contentsname}{\centerline{\bf Contents}}
\renewcommand{\refname}{\centerline{\bf Literature}}

\newcommand{\GP}{\mathcal{GP}}
\newcommand{\E}{\mathbb{E}}
\newcommand{\R}{\mathbb{R}}
\newcommand{\N}{\mathcal{N}}
\newcommand{\cov}{\mbox{cov}}
\newcommand{\Nystrom}{Nystr\"{o}m }
\newcommand{\KL}[2]{\mbox{KL}\left(#1\mbox{ || }#2\right)}
\newcommand{\tr}{\mbox{tr}}
\newcommand{\derivative}[2]{\frac{\partial #1}{\partial #2}}
\newcommand{\sndderivative}[3]{\frac{\partial^2 #1}{\partial #2 \partial #3}}

\newlength{\arrayrulewidthOriginal}
\newcommand{\Cline}[2]{%
  \noalign{\global\setlength{\arrayrulewidthOriginal}{\arrayrulewidth}}%
  \noalign{\global\setlength{\arrayrulewidth}{#1}}\cline{#2}%
  \noalign{\global\setlength{\arrayrulewidth}{\arrayrulewidthOriginal}}}

\newtheorem{definition}{Definition}
\newtheorem{theorem}{Theorem}


\def\vec#1{\mathchoice{\mbox{\boldmath$\displaystyle#1$}}
{\mbox{\boldmath$\textstyle#1$}} {\mbox{\boldmath$\scriptstyle#1$}} {\mbox{\boldmath$\scriptscriptstyle#1$}}}

%\maketitle
\centerline{Lomonosov Moscow State University}

\centerline{Faculty of Computer Science}

\vspace{5 cm}

\centerline{\Large Review of matherials on}

\vspace{1 cm}

\centerline{\Large \bf Gaussian Processes for Machine Learning}

\vspace{6 cm}

\begin{flushright}

Pavel Izmailov
\end{flushright}

\vfill 

\centerline{Moscow,  2016}
\thispagestyle{empty} 
\pagebreak

\section{Theory}

In this section an introduction to Gaussian process theory is provided.

\subsection{Gaussian Process}
	Consider the following definition
	\begin{definition}
		A Gaussian process is a collection of random variables, any finite number of which have a joint Gaussian distribution.
	\end{definition}
	A Gaussian process is completely specified by it's mean function and covariance function. These functions are defined as follows
	\begin{definition}
		Let $f(x)$ be a real-valued Gaussian process. Then the functions
		$$m(x) = \E[f(x)],$$
		$$k(x, x') = \E[(f(x) - m(x)) (f(x') - m(x'))],$$
		are the mean function and the covariance function of the process $f$ respectively. 
	\end{definition}
	
	We will write the Gaussian process as $f(x) \sim \GP(m(x), k(x, x'))$.
	
\subsection{GP-regression}
	Consider the following task. We have a dataset $\{(x_i, f_i) | i = 1, \ldots, n\}$, generated from a Gaussian process $f \sim \GP(m(x), k(x, x'))$, let $x \in \R^d$.  We will denote the matrix comprised of points $x_1, \ldots, x_n$ by $X \in \R^{n \times d}$ and the vector of corresponding values $f_1, ..., f_n$ by $f \in \R^n$. We want to predict the values $f_* \in \R^m$ of this random process at a set of other m points $X_* \in \R^{m \times d}$. The joint distribution of $f$ and $f_*$ is given by
	$$
	\left [ \begin{array}{c} f\\ f_* \end{array} \right ]
	\sim
	\N \left ( 0, \left [\begin{array}{cc} K(X, X) & K(X, X_*)\\ K(X_*, X) & K(X_*, X_*) \end{array} \right] \right ),
	$$
	where $K(X, X) \in \R^{n \times n}$, $K(X, X_*) = K(X^*, X)^T \in \R^{n \times m}$, $K(X^*, X^*) \in \R^{m \times m}$ are the matrices comprised of pairwise values of the covariance function $k$ for the given sets.
	
	The conditional distribution
	
	$$f_* | X_*, X, f \sim \N( \hat m, \hat K ),$$
	where 
	$$\E [f_* | f ] = \hat m = K(X_*, X) K(X, X)^{-1} f,$$
	$$\cov(f_* | f ) = \hat K = K(X_*, X_*) - K(X_*, X)K(X, X)^{-1}K(X, X_*).$$
		
	Thus, predicting the values of the Gaussian process at a new data point requires solving a linear system with a matrix of size $n \times n$ and thus scales as $O(n^3)$.

	\begin{figure}[!h]
		\centering
		\subfloat{
			\scalebox{0.7}{
				\input{../../Code/Experiments/pictures/1dgp-regression.pgf}
			}
		}
		\subfloat{
			\scalebox{0.7}{
	    		\input{../../Code/Experiments/pictures/2dgp-regression.pgf}
			}
		}
		\caption{One and two-dimensional gaussian processes}
		\label{brute_reg_example}
	\end{figure}


	In fig. \ref{brute_reg_example} you can see the examples of one and two-dimensional gaussian-processes, reconstructed from the data. The data points are shown by black `$+$' signs.
	
	\subsubsection{Noisy case}
		Consider the following model. We now have a dataset $\{(x_i, y_i)| i = 1, \ldots n\}$, where $y_i = f(x_i) + \varepsilon$, $\varepsilon \sim \N(0, \sigma_n)$. This means that we only have access to the noisy observations and not the true values of the process at data points. With the notation and logics similar to the one we used it the previous section we can find the conditional distribution for the values $f_*$ of the process at new points $X_*$ in this case:
		$$f_* | y \sim \N( \hat m, \hat K ),$$
		$$\E[f_* | y] = \hat m = K(X_*, X) (K(X, X) + \sigma_n^2 I)^{-1} y,$$
		$$\cov(f_* | y ) = \hat K = K(X_*, X_*) - K(X_*, X)(K(X, X) + \sigma_n^2 I)^{-1}K(X, X_*).$$
		
\subsection{GP-classification}
	\label{gp-classification}
Now we will apply the Gaussian processes to the binary classification problem, which can be described as follows. We have a dataset $\{(x_i, y_i) | i = 1, \ldots, n\}$, where $x_i \in \R^d$, $y_i \in \{-1, 1\}$. We want to predict whether or not a new data point $x_*$ belongs to the positive class, given its coordinates.

We will consider the following model. We will introduce a latent function $f: \R^d \rightarrow \R$ and put a zero-mean GP prior over it. 
$$f \sim \GP(0, k(\cdot, \cdot)).$$
We will then consider the probability of the object $x_*$ belonging to positive class to be equal to $\sigma(f(x_*))$ for the chosen sigmoid function $\sigma$.
$$p(y_* = +1 | x_*) = \sigma(f(x_*)).$$

Note, that the graphical for this model is exactly the same, as for the regression problem and is given in fig. \ref{gp_graphical_model}.

We will use the logistic function $\sigma(z) = (1 + \exp(-z))^{-1}$, however one can use other sigmoid functions as well.

Now inference can be done in two steps. First, for the new data point $x_*$ we should find the conditional distribution of the value of the latent process $f$ at the new data point $x_*$. This can be computed as follows
\begin{equation}
	\label{classification_conditional}
	% p(f_* | X, y, x_*) = \int p(f_* | X, x_*, f) p(f | X, y) df.
	p(f_* | y) = \int p(f_* | f) p(f | y) df.
\end{equation}
Now, the probability that $x_*$ belongs to the positive class is obtained by marginalizing over the latent variable $f_*$.
\begin{equation}
	\label{classification_class_probability}
	% p(y_* = +1 | X, y, x_*) = \int \sigma(f_*) p(f_* | X, y, x_*) df_*.
	p(y_* = +1 | y) = \int \sigma(f_*) p(f_* | y) df_*.
\end{equation}

Unfortunantely, both integrals in (\ref{classification_conditional}) and (\ref{classification_class_probability}) are intractable. Thus, we have to use integral-approximation techniques to estimate the predictive distribution. 

For example, one can use Laplace approximation method, which builds a Gaussian approximation $q(f | X, y)$ to the true posterior $p(f | X, y)$. This approximation is obtained by performing the Taylor expansion of the function $\log p(f | X, y)$ around its maximum $\hat f$. 

Substituting this Gaussian approximation back into (\ref{classification_conditional}) and (\ref{classification_class_probability}), we obtain tractable integrals and can compute the predictive distribution in a closed form. The more detailed derivation of this algorithm and another algorithm, based on Expectation Propagation can be found in \cite{GPinML}.

We will also describe another method for GP-classification below.

Computational complexity of computing the predictive distribution for this method scales as $\bigO(n^3)$.
	
\subsection{Kernel functions}
	To be wrritten.
	
\subsection{Hyper-parameter estimation}
	Bayesian paradigm provides a way of estimating the kernel hyper-parameters of the GP-model through maximizization of the marginal likelihood of the model. Marginal likelihood is given by
	$$p(y | X) = \int p(y | f, X) p(f | X) df,$$
	which is the likelihood, marginalized over the hidden values $f$ of the underlying process.

	For the GP-regression the marginal likelihood can be computed in claused form and is given by
	\begin{equation}
		\label{regression_ml}
		\log p(y | X) = -\frac 1 2 y^{T} (K + \sigma_n^2 I)^{-1} y - \frac 1 2 \log |K + \sigma_n^2 I| - \frac n 2 \log 2 \pi.
	\end{equation}

	For the method, described in section \ref{gp-classification} the marginal likelihood can be computed as follows.
	$$p(y | X) = \int p(y | f, X) p(f | X) df = \int \exp{\Psi(f)} df,$$
	where we use the notation from section \ref{gp-classification}. Using the Taylor expansion, locally near $\hat f$ we have $\Psi(f) \simeq \Psi(\hat f) + \frac 1 2(f - \hat f)^T A (f - \hat f)$, where $A$ is the hessian of $\Psi$ at $\hat f$. Using this approximation we obtain
	$$p(y | X) \simeq q(y | X) = \exp(\Psi(\hat f)) \int \exp( - \frac 1 2 (f - \hat f)^T A (f - \hat f)) df.$$
	This last integral can be calculated analytically to obtain a closed form approximation to the log marginal likelihood. 
	\begin{equation}
		\label{classification_ml}
		\log q(y|X) = -\frac 1 2 \hat f^T K^{-1} \hat f + \log p(y|\hat f) - \frac 1 2 \log|B|,
	\end{equation}
	where 
	$$|B| = |K| \left|- \left. \frac{\partial^2 \log p(y | f)}{\partial f^2} \right|_{f = \hat f} \right|.$$ 

	Using the derived formulas \ref{classification_ml} and \ref{regression_ml} we can find the optimal values of hyper-parameters through maximization of the marginal likelihood of the corresponding model.
	
\subsection{Theoretical perspectives}
	\hspace{0.6cm}To be wrritten.

\pagebreak
\section{Review of existing methods}
	It follows from the discussion above, that full Gaussian process regression scales as $O(n^3)$ and thus cannot be applied to big datasets. In this section we will review several approximate methods, that make Gaussian processes practical.

\subsection{Methods, based on inducing inputs}
	Most of the existing methods are based on introducing a set of $m$ function points that are called inducing inputs. Using these inputs one can make approximate predictions of the values of the hidden process at test points with a complexity of $O(nm^3)$ instead of $O(n^3)$.
	
	Consider the following situation. We have a dataset of $n$ examples $x_i$ with corresponding values $y_i$. We will denote the matrix of pairwise values of the covariance function by $K_{nn}$. Now we introduce a set of $m$ inducing inputs. We will denote the corresponding covariance matrix by $K_{mm}$ and the matrices of covariances between the inducing points and training points by $K_{nm}$ and $K_{mn}$. We will denote the vectors, comprised of noisy and true function values $y_i$ and $f_i$ at training points by $y$ and $f$ respectively. We will also introduce a distribution $q(u)$ over the hidden function values $u$ at the inducing inputs.
	
	It's easy to see, that
	$$p(y|f) = \N (y|f, \sigma_n I),$$
	$$p(f|u) = \N (f|K_{nm} K_{mm}^{-1}u, \tilde K),$$
	$$p(u) = \N(u|0, K_{mm}),$$
	where $\tilde K = K_{nn} - K_{nm} K_{mm}^{-1} K_{mn}.$
		
	\subsubsection{Variational learning of inducing points}
		\label{Titsias}
		
		The method discussed here was introduced in \cite{Titsias}. This method provides a way to find the optimal positions of the inducing points, as well as the optimal distribution of the process value at these points.

		Let $z$ denote a vector comprized of the process values at some new points. We can calculate the predictive distribution at these points as follows
		$$p(z|y) = \int p(z|f) p(f|y) df.$$
		Let's fix the inducing point positions $x_1, \ldots, x_m$. As above, $u$ is the vector compised of the process values at these points. We can rewrite the above equation
		\begin{equation}
			\label{predictive1}
			p(z|y) = \iint p(z|u, f) p(f| u, y) p(u|y)df du,
		\end{equation}
		% $$p(z | y) = \iint p(z | u, f) p(f | u, y)df du,$$
		as $p(z|u, f, y) = p(z|u, f)$. 

		Suppose for a moment, that $u$ is a sufficient statistics for the parameter $f$ in the scence that $z$ and $f$ are conditionally independent given $u$. Then we have 
		$$p(z|f, u) = \frac {p(z, f|u)} {p(f|u)} = \frac {p(z | u) p(f | u)}{p(f|u)} = p(z|u),$$
		$$p(z|y, u) = \frac {p(z, y, u)}{p(y, u)} = \frac {\int p(y|f)p(f, z, u) du}{\iint p(y|f) p(f, z, u) df dz} = \frac {\int p(y|f) p(z|u) p(u|f) p(f)df}{\iint p(y|f) p(z|u) p(u|f) p(f)df dz} = $$
		$$= \frac {\int p(y|f)p(f)p(u|f)df \cdot p(z|u)} {\int p(y|f)p(f)p(u|f)df \cdot \int p(z|u) dz} = \frac{\int p(y, f) p(u|f) df} {\int p(y, f) p(u|f) df} p(z|u) = p(z|u).$$

		So, $p(z|y, u) = p(z|u)$. If we set the points, corrwsponding to the process values $z$, to the traing points, we will have $z = f$, and thus $p(f|y, u) = p(f|u)$.

		Substituting these formulas into (\ref{predictive1}) we achieve
		$$q(z) = p(z|y) = \iint p(z|u) p(f|u) p(u|y)df du = \iint p(z|u) p(u|y) du = $$
		\begin{equation}
			\label{predictive2}
			= \int p(z|u)\varphi(u) du  = \int q(z, u) du, 
		\end{equation}
		where $\varphi(u) = p(u|y)$, $q(z, u) = p(z|u)\varphi(u)$.

		In practice however it's difficult to guarantee that $u$ is a sufficient statistics. Thus we can only expect $q(z)$ to be an approximation to $p(z|y)$. In such case we can choose $\varphi(u)$ to be a variational distribution, where in general $\varphi(u) \ne p(u | y)$. We will consider $\varphi(u)$ to be Gaussian with a mean vector $\mu$ and covariance matrix $\Sigma$.

		By using the eq. (\ref{predictive2}) we can calculate the approximate posterior GP mean at point $x$ and covariance at points $x, x'$
		$$\E[z(x)] = K_{xm} K_{mm}^{-1} \mu,$$ 
		$$\cov(z(x), z(x')) = k(x, x') - K_{xm} K_{mm}^{-1} K_{mx'} + K_{xm} A K_{mx'},$$
		where $A = K_{mm}^{-1} \Sigma K_{mm}^{-1}$.

		Now we have to specify a way to find the variational distribution parameters $\mu$ and $\Sigma$, and the inducing input positions $X_m$ and a way to optimize the kernel hyper-parameters. 
		% In order to do so, we will form the variational distribution $q(f)$ and the exact posterior $p(f|y)$ on the training function values, and then minimize the distance between this two distributions. Equivalently, we can minimize a distance, between the augmented true posterior $p(f, u|y)$ and $q(f, u)$.
		In order to do so, we will form the variational distribution $q(f, u)$ and the exact posterior $p(f, u|y)$ on the training function values and the values at the inducing points, and then minimize the KL-divergence between these two distributions. This minimization is equivalently expressed as the maximization of the following lower bound of the true marginal likelihood:
		$$F_V(X_m, \varphi) = \iint p(f|u) \varphi(u) \log \frac{p(y|f) p(u)}{\varphi(u)} df du.$$
		This bound can be optimized analytically with respect to $\phi$. The optimal distribution $\varphi(u) \sim \N(u|\hat u, \Lambda^{-1})$, where
		$$\Lambda = \frac 1 {\sigma_n} K_{mm}^{-1} K_{mn} K_{nm} K_{mm}^{-1} + K_{mm}^{-1},$$
		$$\hat u = \frac 1 {\sigma_n} \Lambda^{-1} K_{mm}^{-1} K_{mn} y.$$
		Substituting the optimal values of variational parameters into the $F_V$ we obtain the following bound
		$$F_V(X_m) = \log \N(y|0, \sigma_n^2 I + K_{nm} K_{mm}^{-1} K_{mn}) - \frac 1 {2\sigma_n^2} \tr(\tilde K).$$

		Another derivation of this lower bound is provided in section (\ref{svi}).

		The bound $F_V(X_m)$ is computed in $o(nm^2)$ time. Now we will calculate it's gradient in order to be able to maximize it with respect to $X_m$ and kernel hyper-parameters. We will denote $B = \sigma_n^2 I + K_{nm} K_{mm}^{-1} K_{mn}$. Then
		$$F_V(X_m, \theta, \sigma_n) = -\frac 1 2 \left(n \log 2\pi + \log |B| + y^T B^{-1} y + \frac 1 {\sigma_n^2} \tr(\tilde K)\right),$$
		$$\derivative{F_V}{\theta} = \frac 1 2 \left( -\tr \left(B^{-1} \derivative{B}{\theta}\right) + y^T B^{-1} \derivative{B}{\theta} B^{-1} y - \right.$$    
		$$- \left. \frac 1 {\sigma_n^2} \tr\left(\derivative{K_{nn}}{\theta} - \left(\derivative{K_{nm}}{\theta}K_{mm}^{-1} - K_{nm} K_{mm}^{-1} \derivative{K_{mm}}{\theta}K_{mm}^{-1}\right) K_{mn} - K_{nm} K_{mm}^{-1} \derivative{K_{mn}}{\theta}\right)\right),$$
		where
		$$\derivative{B}{\theta} = \left(\derivative{K_{nm}}{\theta}K_{mm}^{-1} - K_{nm} K_{mm}^{-1} \derivative{K_{mm}}{\theta}K_{mm}^{-1}\right) K_{mn} +  K_{nm} K_{mm}^{-1} \derivative{K_{mn}}{\theta}.$$

		We can rewrite
		$$\derivative{F_V}{\theta} = \frac 1 2 \left( -\tr \left(B^{-1} \derivative{B}{\theta}\right) + y^T B^{-1} \derivative{B}{\theta} B^{-1} y - \frac 1 {\sigma_n^2} \tr \left(\derivative {K_{nn}} {\theta} - \derivative {B}{\theta}\right) \right).$$

		Now we can optimize $F_V$ with respect to kernel hyper-parameters. Similarly, we can take derivatives with respect to $X_m$ and $\sigma_n$ and opptimize $F_V$ with respect to them as well.

		However, if we compute $F_v$ and it's derivatives as they are, it takes $O(n^3)$ time which is not faster, than recovering the full Gaussian process. So, we have to rewrite these values in a form that allows for faster computation.

		First of all, let's deal with $\log|B|$ and $B^{-1}$. Using the matrix determinant lemma we obtain
		$$|B| = |\sigma_n^2 I + K_{nm} K_{mm}^{-1} K_{mn}| = \frac{\left|K_{mm} + \cfrac{K_{mn} K_{nm}}{\sigma_n^2}\right| \sigma_n^2}{|K_{mm}|}.$$
		So, denoting $A = K_{mm} + \cfrac{K_{mn} K_{nm}}{\sigma_n^2}$, we obtain
		$$\log |B| = \log |A| + 2 \log \sigma_n - \log |K_{mm}|.$$
		Note tha this is computed in $O(n m^2)$ instead of $O(n^3)$.

		Using the Woodbury identity, we obtain
		$$B^{-1} = (\sigma_n^2 I + K_{nm} K_{mm}^{-1} K_{mn})^{-1} = \frac I {\sigma_n^2} - \frac{K_{nm} A^{-1} K_{mn}}{\sigma^{4}},$$
		which allows for computing $y^T B^{-1} y$ in $O(n m)$.

		Similarly, we can compute the gradient in $O(nm^2)$. In order to do so, we need to rewrite every trace $\tr(M_{nm} M_{mm} M_{mn})$, where $M_{kl} \in \R^{k \times l}$, in the form $\tr(M_{mm} M_{mn} M_{nm})$, which is computed in $O(nm^2)$, and use the derived formulas for $B^{-1}$.

		Now let's try to compute the second order derivatives.
		$$\frac{\partial^2 F_V} {\partial \theta_j \partial\theta_i} = \derivative{}{\theta_j} \left(\derivative{F_V}{\theta_i}\right) = \frac 1 2 \derivative{}{\theta_j} \left( -\tr \left(B^{-1} \derivative{B}{\theta_i}\right) + y^T B^{-1} \derivative{B}{\theta_i} B^{-1} y - \frac 1 {\sigma_n^2} \tr \left(\derivative {K_{nn}} {\theta_i} - \derivative {B}{\theta_i}\right)\right) = $$
		$$ = \frac 1 2 \left(\tr\left( B^{-1} \derivative{B}{\theta_j} B^{-1}\derivative{B}{\theta_i} - B^{-1} \sndderivative{B}{\theta_j}{\theta_i}\right) + y^T \left(B^{-1} \sndderivative{B}{\theta_j} {\theta_i} B^{-1}  - 2 B^{-1} \derivative{B}{\theta_j} B^{-1} \derivative{B}{\theta_i} B^{-1} \right) y \right. - $$
		$$ \left.- \frac 1 {\sigma_n^2} \tr\left(\sndderivative{K_{nn}}{\theta_j}{\theta_i} - \sndderivative{B}{\theta_j}{\theta_i}\right)\right),$$
		where
		$$\sndderivative{B}{\theta_j}{\theta_i} = \derivative{}{\theta_j} \left(\derivative{K_{nm}}{\theta_i}K_{mm}^{-1} K_{mn} - K_{nm} K_{mm}^{-1} \derivative{K_{mm}}{\theta_i}K_{mm}^{-1} K_{mn} +  K_{nm} K_{mm}^{-1} \derivative{K_{mn}}{\theta_i}\right) = $$

		$$ = \sndderivative{K_{nm}}{\theta_j}{\theta_i}K_{mm}^{-1} K_{mn} + K_{nm} K_{mm}^{-1}\sndderivative{K_{mn}}{\theta_j}{\theta_i}  - \derivative{K_{nm}}{\theta_i}K_{mm}^{-1} \derivative{K_{mm}}{\theta_j}K_{mm}^{-1}K_{mn} - $$
		
		$$- K_{nm} K_{mm}^{-1} \derivative{K_{mm}}{\theta_j}K_{mm}^{-1}\derivative{K_{mn}}{\theta_i} + \derivative{K_{nm}}{\theta_j}K_{mm}^{-1} \derivative{K_{mn}}{\theta_i} + \derivative{K_{nm}}{\theta_i} K_{mm}^{-1} \derivative{K_{mn}}{\theta_j} $$

		$$- \derivative{K_{nm}}{\theta_j} K_{mm}^{-1} \derivative{K_{mm}}{\theta_i}K_{mm}^{-1} K_{mn} + K_{nm} K_{mm}^{-1}\derivative{K_{mm}}{\theta_j} K_{mm}^{-1}\derivative{K_{mm}}{\theta_i}K_{mm}^{-1} K_{mn}$$

		$$ - K_{nm} K_{mm}^{-1} \sndderivative{K_{mm}}{\theta_j}{\theta_i}K_{mm}^{-1} K_{mn} + K_{nm} K_{mm}^{-1} \derivative{K_{mm}}{\theta_i}K_{mm}^{-1}\derivative{K_{mm}}{\theta_j}K_{mm}^{-1} K_{mn} - $$

		$$- K_{nm} K_{mm}^{-1} \derivative{K_{mm}}{\theta_i}K_{mm}^{-1} \derivative{K_{mn}}{\theta_j}.$$


	\pagebreak
	\subsubsection{Stochastic variational inference}
		In this subsection we describe a method for maximizing the lower bound (\ref{main_elbo}) in case of the GP-regression problem, which was proposed in \cite{BigData}. While the method, described in the previous section is much faster then the full GP-regression, it's complexity is still rather big. We could try, to reduce the time consumption of optimizing the lower bound, by using stochastic optimization methods. However, the function in the right hand side of (\ref{titsias_elbo}) does not have a form of sum over objects, and thus it's not clear, how to apply the stochastic methods.

However, the original bound from (\ref{main_elbo}) does have a form of sum over objects, and we can thus apply stochastic methods to it. In the regression case the bound looks like
$$\log p(y) \ge \sum_{i = 1}^{n} \left( \log \N(y_i | k_i^T K_{mm}^{-1} \mu, \sigma_n^2) - \frac 1 {2 \sigma_n^2} \tilde K_{ii} - \frac 1 2 \tr (\Sigma \Lambda_i) \right) - $$
$$ -\frac 1 2 \left (\log \frac {|K_{mm}|} {|\Sigma|} - m + \tr(K_{mm}^{-1} \Sigma) + \mu^T K_{mm}^{-1} \mu \right).$$

In the \lstinline{svi} method, we directly optimize this ELBO with respect to both variational parameters and kernel hyper-parameters in a stochastic way. The authors of the method suggest to use the stochastic gradient descent with natural gradients for the variational parameters and usual gradients for kernel hyper-parameters.

The natural gradients are the gradients with respect to the natural parameters of an exponential family of distributions. These gradients are considered to be effective in the case of optimization with respect to probability distribution parameters, because they use symmetrized $\mbox{KL}$ divergence between the distributions instead of usual distance between distribution parameters as a distance metric. For more information about natural gradients see for example \cite{ExpFamilyGeom}.

The complexity of computing a stochastic update of the variational and kernel parameters is independent of $n$ and scales as $\bigO(m^3)$. Thus, the stochastic optimization might give this method advantage against the \lstinline{vi} method. However, for big data problems the number of required inducing points $m$ is usually quite big. The number of parameters we have to optimize scales as $\bigO(m^2)$, which makes the optimization problem of the \lstinline{svi} method much harder then the one we have to solve in the \lstinline{vi} method. We will compare the two methods in the experiments section.


	\subsection{Stochastic variational inference for classification}
		The method described here was proposed in \cite{SVIclassification}. We could also apply the svi method to the classification problem. In order to do so, we will first rederive the ELBO from (\ref{L3}).

We will use the augmented model for the data.
$$p(y, f, u) = p(y | f) p(f | u) p(u).$$
\begin{figure}[!h]
	\centering
	\subfloat{
		\scalebox{0.7}{
			\begin{tikzpicture}
	\input{base_graphical_model.tikz}
\end{tikzpicture}


		}
	}
	\hspace{1cm}
	\subfloat{
		\scalebox{0.7}{
			\begin{tikzpicture}
	\tikzstyle{u} = [circle, draw, fill=red!50, minimum size=1.2cm, text width=0.8cm, align=center, font=\large]
	\input{base_graphical_model.tikz}
	\pgfmathsetmacro{\layerpos}{\step/2}
	\node[u] (inputs) at (1.25, 6) {$u$};

	\foreach \to in {f_1, f_2, f_n}
		\draw[edge] (inputs) -- (\to);
\end{tikzpicture}


		}
	}
	\caption{Graphical models for standard and inducing-point gaussian process classification}
\end{figure}

Applying the standard variational lower bound to this model, we obtain
$$\log p(y) \ge \E_{q(u, f)} \log \frac {p(y, u, f)}{q(u, f)} = \E_{q(u, f)}\log p(y | f) - \KL{q(u, f)} {p(u, f)}.$$
Our model implies $\E_{q(u, f)} \log p(y | f) = \E_{q(f)} \log p(y | f)$, where $q(f)$ is the marginal of $q(u, f)$.

We will consider the variational distributions of the following form:
$$q(u, f) = p(f | u) q(u),$$
where $q(u) \sim \N(u|\mu, \Sigma)$. This implies $q(f)$
$$q(f) = \int p(u | f) q(u) du = 
\N(f| K_{nm} K_{mm}^{-1} \mu, K_{nn} + K_{nm} K_{mm}^{-1}(\Sigma - K_{mm}) K_{mm}^{-1} K_{mn}).$$

Now, consider the KL-divergence in the lower bound we've devised.
$$\KL{q(u, f)} {p(u, f)} = \KL{q(u) p(f|u)} {p(u) p(f|u)} = \KL{q(u)} {p(u)}.$$

Finally, the lower bound is
\begin{equation}\label{sviELBO}
	\log p(y) \ge \E_{q(f)} \log p(y | f) - \KL{q(u)} {p(u)} = $$ $$ = \sum_{i = 1}^{n} \E_{q(f_i)} \log p(y_i | f_i) - \KL{q(u)} {p(u)},
\end{equation}
where 
$$q(f_i) = \N(f_i | k_i^T K_{mm}^{-1} \mu, K_{ii} + k_i^T K_{mm}^{-1} (\Sigma - K_{mm}) K_{mm}^{-1} k_{i}) = \N(f_i | m_i, S_i^2)$$
Note, that this lower bound is exactly the lower bound from (\ref{L3}), but now, we've derived it in a more general setting.

% Note, that 
% $$\E_{q(f_i)} \log p(y_i|f_i) = \E_{\N(f_i| m_i, S_i)} \log p(y_i|f_i) = \E_{\N(t | 0, 1)} \log p(y_i | (t \sqrt{S_i} + m_i))$$

Substituting the distributions $q$ and $p$ back into the (\ref{sviELBO}) we obtain

% $$\log p(y) \ge \sum_{i=1}^n \E_{\N(t | 0, 1)} \log p(y_i | t \sqrt{S_i} + m_i) -$$    
$$\log p(y) \ge \sum_{i=1}^n \E_{\N(f_i | m_i, S_i^2)} \log p(y_i | f_i) -$$
\begin{equation}\label{explicit_svi_elbo}
	-\frac 1 2 \left (\log \frac {|K_{mm}|} {|\Sigma|} - m + \tr(K_{mm}^{-1} \Sigma) + \mu^T K_{mm}^{-1} \mu \right) = L_3(\mu, \Sigma, \theta).
\end{equation}

Now we can maximize this lower bound with respect to variational parameters $\mu$, $\Sigma$ and covariance hyper-parameters $\theta$.

Let's find the derivatives of (\ref{explicit_svi_elbo}).

% $$\derivative{L_3}{\mu} = \sum_{i=1}^n \derivative{}{\mu} \E_{\N(t | 0, 1)} \log p(y_i | t \sqrt{S_i} + m_i) + K_{mm}^{-1} \mu = $$
$$\derivative{L_3}{\mu} = \sum_{i=1}^n \derivative{}{\mu} \left(\E_{\N(f_i | m_i, S_i^2)} \log p(y_i | f_i)\right) - K_{mm}^{-1} \mu = $$
$$ = \sum_{i=1}^n \derivative{}{m_i} \left(\E_{\N(f_i | m_i, S_i^2)} \log p(y_i | f_i)\right) \derivative{m_i}{\mu} - K_{mm}^{-1} \mu,$$
$$ = \sum_{i=1}^n \E_{\N(f_i | m_i, S_i^2)}\left[ \derivative{}{f_i} \log p(y_i | f_i)\right] \derivative{m_i}{\mu} - K_{mm}^{-1} \mu,$$

$$\derivative{L_3}{L_{\Sigma}} = \sum_{i=1}^n \derivative{}{S_i^2}\left( \E_{\N(f_i | m_i, S_i^2)} \log p(y_i | f_i)\right) \derivative{S_i^2}{L_{\Sigma}} + \hat L - K_{mm}^{-1} L_{\Sigma} = $$
$$= \frac 1 2 \sum_{i=1}^n \E_{\N(f_i | m_i, S_i^2)}\left[ \derivative{^2}{f_i^2} \log p(y_i | f_i)\right] \derivative {S_i^2}{L_{\Sigma}} + \hat L - K_{mm}^{-1} L_{\Sigma},$$
where
$$\hat L = 
\left(
\begin{array}{cccc}
\frac 1 {(L_{\Sigma})_{11}} & 0 & \ldots & 0\\
0 & \frac 1 {(L_{\Sigma})_{22}} & \ldots & 0\\
\ldots & \ldots & \ldots & \ldots\\
0 & 0 & \ldots & \frac 1 {(L_{\Sigma})_{mm}} \\
\end{array}   
\right) $$


$$\derivative{L_3}{\theta} = \derivative{}{\theta} \left(\sum_{i=1}^n \E_{\N(t | m_i, S_i^2)}  \log p(y_i | f_i)\right) -$$
$$- \frac 1 2 \tr\left(K_{mm}^{-1} \frac{\partial K_{mm}}{\partial \theta}\right) + \frac 1 2 \tr\left(\Sigma K_{mm}^{-1} \frac{\partial K_{mm}}{\partial \theta} K_{mm}^{-1} \right)+ \frac 1 2 \mu^T K_{mm}^{-1} \frac{\partial K_{mm}}{\partial \theta} K_{mm}^{-1}\mu.$$

Note that
$$\derivative{}{\theta} \left(\E_{\N(f_i | m_i, S_i^2)} \log p(y_i | f_i)\right) = \derivative{}{m_i}\left(\E_{\N(t | m_i, S_i^2)}  \log p(y_i | f_i) \right)\derivative{m_i}{\theta} + \derivative{}{S_i^2}\left(\E_{\N(f_i | m_i, S_i^2)}  \log p(y_i | f_i) \right)\derivative{S_i^2}{\theta} = $$
$$ = \E_{\N(f_i | m_i, S_i^2)} \left[\derivative{}{f_i} \log p(y_i | f_i) \right] \derivative{m_i}{\theta} + \frac 1 2 \E_{\N(f_i | m_i, S_i^2)} \left[\derivative{^2}{f_i^2} \log p(y_i | f_i) \right] \derivative{S_i^2}{\theta}$$

In order to compute the expectations in the derivatives, we can use the integral approximation techniques, and Gauss-Hermite quadrature in particular.

We will use logistic likelihood
$$\log p(y_i | f_i) =  - \log(1 + \exp( - y_i f_i)).$$

Then
$$\derivative{}{f_i} \log p(y_i | f_i) = \frac{y_i}{1 + \exp(y_i f_i)},$$
$$\derivative{^2}{f_i^2} \log p(y_i | f_i) = - \frac{y_i^2 \exp(y_i f_i)}{(1 + \exp(y_i f_i))^2}.$$

Now,
$$m_i = k_i^T K_{mm}^{-1} \mu,$$
$$\derivative{m_i}{\mu} = K_{mm}^{-1} k_i$$
$$\derivative{m_i}{\theta} = \derivative{k_i}{\theta}^T K_{mm}^{-1} \mu - k_i^T K_{mm}^{-1} \derivative{K_mm}{\theta} K_{mm}^{-1} \mu.$$
Finally, 
$$S_i^2 = K_{ii} + k_i^T K_{mm}^{-1} (L_\Sigma L_\Sigma^T - K_{mm}) K_{mm}^{-1} k_{i},$$
$$\derivative{S_i^2}{L_{\Sigma}} = 2 K_{mm}^{-1} k_i k_i^T K_{mm}^{-1} L_{\Sigma},$$
$$\derivative{S_i^2}{\theta} = \derivative{K_{ii}}{\theta} + 2 \derivative{k_i}{\theta}^T K_{mm}^{-1} (L_\Sigma L_\Sigma^T - K_{mm}) K_{mm}^{-1} k_{i}$$
$$ - 2 k_i^T K_{mm}^{-1} \derivative{K_{mm}}{\theta} K_{mm}^{-1} (L_\Sigma L_\Sigma^T - K_{mm}) K_{mm}^{-1} k_{i} - k_i^T K_{mm}^{-1} \derivative{K_{mm}}{\theta} K_{mm}^{-1} k_{i}.$$

The final formulas for the derevatives are
$$\derivative{L_3}{\mu} = \sum_{i=1}^n \E_{\N(f_i | m_i, S_i^2)}\left[\frac{y_i}{1 + \exp(y_i f_i)}\right] K_{mm}^{-1} k_i + K_{mm}^{-1} \mu,$$

$$\derivative{L_3}{L_{\Sigma}} = \sum_{i=1}^n \E_{\N(f_i | m_i, S_i^2)}\left[- \frac{y_i^2 \exp(y_i f_i)}{(1 + \exp(y_i f_i))^2} \right] K_{mm}^{-1} k_i k_i^T K_{mm}^{-1} L_{\Sigma} + \hat L - K_{mm}^{-1} L_{\Sigma},$$

$$\derivative{L_3}{\theta} = \sum_{i=1}^n \left[\E_{\N(f_i | m_i, S_i^2)} \left[\frac{y_i}{1 + \exp(y_i f_i)}\right] \left(\derivative{k_i}{\theta}^T K_{mm}^{-1} \mu - k_i^T K_{mm}^{-1} \derivative{K_mm}{\theta} K_{mm}^{-1} \mu \right)\right.$$
$$ + \frac 1 2 \E_{\N(f_i | m_i, S_i^2)} \left[- \frac{y_i^2 \exp(y_i f_i)}{(1 + \exp(y_i f_i))^2}\right] \left( \derivative{K_{ii}}{\theta} + 2 \derivative{k_i}{\theta}^T K_{mm}^{-1} (L_\Sigma L_\Sigma^T - K_{mm}) K_{mm}^{-1} k_{i} \right.$$
$$ \left.\left.- 2 k_i^T K_{mm}^{-1} \derivative{K_{mm}}{\theta} K_{mm}^{-1} (L_\Sigma L_\Sigma^T - K_{mm}) K_{mm}^{-1} k_{i} - k_i^T K_{mm}^{-1} \derivative{K_{mm}}{\theta} K_{mm}^{-1} k_{i} \right)\right]- $$
$$- \frac 1 2 \tr\left(K_{mm}^{-1} \frac{\partial K_{mm}}{\partial \theta}\right) + \frac 1 2 \tr\left(\Sigma K_{mm}^{-1} \frac{\partial K_{mm}}{\partial \theta} K_{mm}^{-1} \right)+ \frac 1 2 \mu^T K_{mm}^{-1} \frac{\partial K_{mm}}{\partial \theta} K_{mm}^{-1}\mu.$$





	\pagebreak

\section{Experiments}
	In this section the results of the numerical experiments are provided. All of the provided plots has a title, that tells the number of training points $n$, the number of features $d$ and the number of inducing points $m$. The title also tells the name of the dataset.

	The methods were compared on variaus datasets. Some of them are generated from a gaussian process and others are real. The $R^2$-score on a test set was used as a quality metric. 

	The squared exponential kernel was used in all the experiments.

	\subsection{Variations of the stochastic variational inference method}
		In this section we compare several variations of the stochastic variational inference method.

		The first variation is denoted by \lstinline{svi-natural}. It is the method described in \cite{BigData}. It uses stochastic gradient descent with natural gradients for minimizing the ELBO with respect to the variational parameters, and usual gradients with respect to kernel hyperparameters.

		The methods \lstinline{svi-L-BFGS-B} and \lstinline{svi-FG} use the full (non-stochastic) ELBO from the same article \cite{BigData} and minimize it with L-BFGS-B and gradient descent respectively. These methods use Cholesky factorization (see \ref{svi}) for the variational parameters.

		Finally, the \lstinline{svi-SAG} method to minimize the ELBO. This method also uses Cholesky factorization.

		We will compare the methods on datasets, generated from some gaussian process and on real data. 

		% The generated dataset consisted of $500$ train points with $2$ features. $100$ inducing inputs were used 

		The results on small and medium datasets are shown in fig. \ref{svi_small} and fig. \ref{svi_medium}.

		\begin{figure}[h!]
			\centering

			\subfloat{
				\scalebox{0.75}{
					\input{../../Code/Experiments/Plots/svi_variations/small_generated.pgf}
				}
			}
			\subfloat{
				\scalebox{0.75}{
		    		\input{../../Code/Experiments/Plots/svi_variations/small_real.pgf}
				}
			}
			\vspace{0.1cm}
			\caption{Svi methods' performance on small datasets}
			\label{svi_small}
		\end{figure}


		\begin{figure}[h!]
			\centering
			\subfloat{
				\scalebox{0.75}{
					\input{../../Code/Experiments/Plots/svi_variations/medium_generated.pgf}
				}
			}
			\subfloat{
				\scalebox{0.75}{
		    		\input{../../Code/Experiments/Plots/svi_variations/medium_real.pgf}
				}
			}
			\label{svi_medium}
			\caption{Svi methods' performance on medium datasets}
		\end{figure}

\subsection{Comparison of stochastic and non-stochastic variational inference methods}
	In this section we compare the \lstinline{vi-means} method with \lstinline{svi-L-BFGS-B}. The \lstinline{vi-means} method is a variation of the method, described in section \ref{Titsias}. It does not optimize for the inducing point positions and does uses \lstinline{L-BFGS-B} to maximize the ELBO.

	\begin{figure}[!h]
		\centering
		\subfloat{
			\scalebox{0.75}{
				\input{../../Code/Experiments/Plots/vi_vs_svi/small_real.pgf}
			}
		}
		\subfloat{
			\scalebox{0.75}{
	    		\input{../../Code/Experiments/Plots/vi_vs_svi/medium_real.pgf}
			}
		}
		
		\caption{Method's performance on small and medium datasets}
	\end{figure}

	\begin{figure}[!h]
		\centering
		\subfloat{
			\scalebox{0.75}{
		    	\input{../../Code/Experiments/Plots/vi_vs_svi/big_real.pgf}
			}
		}
		\caption{Method's performance on a big dataset}
	\end{figure}

	We can see, that \lstinline{vi-means} beats it's oponent in all the experiments. One could expect these results, because \lstinline{vi-means} optimizes the exact same functional as it's oponent, but it uses exact optimal values for some of the parameters. Thus, on moderate problems the \lstinline{vi-means} method beats all the discussed \lstinline{svi} variations.

	Finally, we will compare \lstinline{vi-means} with stochastic \lstinline{svi-natural} an \lstinline{svi-SAG} on a big dataset. The results can be found in fig. \ref{visvi_big}.

	\begin{figure}[!h]
		\centering
		\subfloat{
			\scalebox{0.73}{
				\input{../../Code/Experiments/Plots/vi_vs_svi/1e5_sg_lbfgs.pgf}
			}
		}
		\caption{vi and svi methods comparison on a big dataset}
		\label{visvi_big}
	\end{figure}

\subsection{Variations of variational inference method}
	In this section we compare several variations of the stochastic variational inference method. The method itself is described in section \ref{Titsias}. We compare two different optimization methods for minimizint the Titsias's ELBO.

	The first variation is denoted by \lstinline{means-PN}. It uses Projected-Newton method for minimizing the ELBO. The second variation is denoted by \lstinline{means-L-BFGS-B} and uses L-BFGS-B optimization method.

	The \lstinline{means-PN} uses finite-difference approximation of the hessian. It also makes hessian-correction in order to make it simmetric positive-definite.

	We compare the methods on several different datasets. The results on a small and medium datasets can be found in fig. \ref{vi_small}. The results on a biger dataset can be found in fig. \ref{bi_big}

	\begin{figure}[!h]
		\centering
		\subfloat{
			\scalebox{0.75}{
				\input{../../Code/Experiments/Plots/vi_variations/small_real.pgf}
			}
		}
		\subfloat{
			\scalebox{0.75}{
	    		\input{../../Code/Experiments/Plots/vi_variations/medium_real.pgf}
			}
		}

		\label{vi_small}
		\caption{Method's performance on small and medium datasets}
	\end{figure}
	\begin{figure}[!h]
		\centering
		\subfloat{
			\scalebox{0.75}{
				\input{../../Code/Experiments/Plots/vi_variations/big_real.pgf}
			}
		}
		\subfloat{
			\scalebox{0.75}{
				\input{../../Code/Experiments/Plots/vi_variations/huge_real.pgf}
			}
		}
		\label{vi_big}
		\caption{Method's performance on a bigger dataset}
	\end{figure}

\pagebreak
\begin{thebibliography}{99}

\bibitem{Titsias}
Titsias M. K. (2009).  Variational Learning of Inducing Variables in Sparse Gaussian
Processes.  In: {\it International Conference on Artificial Intelligence and Statistics}, pp.~567–574.

\bibitem{BigData}
Hensman J., Fusi N., Lawrence D. (2013).  Gaussian Processes for Big Data.  In: {\it Proceedings of the Twenty-Ninth Conference on Uncertainty in Artificial Intelligence}.

\bibitem{SVIclassification}
Hensman J., Matthews G., Ghahramani Z. (2015). Scalable Variational Gaussian Process Classification.  In: {\it Proceedings of the Eighteenth International Conference on Artificial Intelligence and Statistics}.

\end{thebibliography}	
\end{document}