As a result of the experiments, we can make several conclusions.

First of all, the inducing input methods proved to be effective. They allow applying the gaussian process framework to the big data problems, which can not be done with the standard methods. 

The \lstinline{vi} method seems to work better than the \lstinline{svi} method even for big data problems, despite the fact, that one can not use stochastic optimization with it. One of the reason for that is that the optimization problem itself in the \lstinline{vi} method is much easier than the one, that has to be solved in the \lstinline{svi} method, and using stochastic optimization does not really help.

The stochastic average gradient optimization method for the \lstinline{svi} method does not beat the stochastic gradient descent with natural gradietns. This implies, that using the natural gradients instead of the usual gradients in the \lstinline{svi} method provides benefits.