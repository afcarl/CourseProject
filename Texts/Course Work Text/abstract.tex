\centerline{\bf Abstract}

Gaussian processes provide an elegant and effective approach to learning in kernel machines. This approach leads to a highly interpretable model and allows using the bayesian framework for model adaptation and incorporating the prior knowledge about the problem. Unfortunately, the standard methods for GP-regression and GP-classification scale as $\bigO(n^3)$, where $n$ is the size of the dataset, which makes them inapplicable for big data problems. In this work we describe two modern methods for GP-regression and one for GP-classification, which attempt to solve this problem, and provide their experimental comparison.