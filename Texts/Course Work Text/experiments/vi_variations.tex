\begin{figure}[t!]
	\centering
	\subfloat{
		\scalebox{0.75}{
			\input{../../Code/Experiments/Plots/vi_variations/small_real.pgf}
		}
	}
	\subfloat{
		\scalebox{0.75}{
    		\input{../../Code/Experiments/Plots/vi_variations/medium_real.pgf}
		}
	}

	% \subfloat{
	% 	\scalebox{0.75}{
	% 		\input{../../Code/Experiments/Plots/vi_variations/big_real.pgf}
	% 	}
	% }
	% \subfloat{
	% 	\scalebox{0.75}{
	% 		\input{../../Code/Experiments/Plots/vi_variations/huge_real.pgf}
	% 	}
	% }
	\caption{ \lstinline{vi} method variations on different datasets}
	\label{vi_results}
\end{figure}

In this section we compare two optimization methods for the \lstinline{vi-means} method.

The first variation is denoted by \lstinline{Projected Newton}. It uses projected Newton method for minimizing the ELBO (\ref{titsias_elbo}). The second variation is denoted by \lstinline{means-L-BFGS-B} and uses L-BFGS-B optimization method.

\lstinline{Projected Newton} method uses finite-difference approximation of the hessian. It also makes hessian-correction in order to make it symmetric and positive-definite. The optimization method itself makes a Newton step and then projects the result to the feasible set in the metric, determined by the hessian. For more information about the method see for example \cite{ProjNewton}.

The time complexity of one iteration for both projected Newton method and L-BFGS-B is $\bigO(nm^2)$. In the projected Newton method we have to compute the hessian matrix of the ELBO with respect to covariance hyper-parameters. In case of squared exponential covariance function the time, needed to compute the hessian, is twice the time, needed to compute the gradient.

The results are provided in fig. \ref{vi_results}. In the provided experiments projected Newton method beats L-BFGS-B. However, the results are close and on different datasets L-BFGS-B beats projected Newton. We need to perform further experiments in order to find out whether using second order optimization provides benefits in the \lstinline{vi} method.

In further experiments we use the L-BFGS-B method because it was more stable in general in our experiments.