In this section we compare the methods, described above, for both regression and classification problems. We will compare the methods, that use inducing points with the standard methods and also compare the different versions of \lstinline{vi} and \lstinline{svi} methods to each other.

All of the plots, apart from the plots in the next subsection, have title of the following format.
$$\mbox{[name of the dataset]}, n = \mbox{[number of objects in the training set]},$$
$$d = \mbox{[number of features]}, m = \mbox{[number of inducing inputs]}$$

The plots in the next subsection do not have the $m$ in the title, because this parameter is not fixed in the corresponding experiments. If the name of the dataset is ``generated'', it means that the dataset was sampled from some Gaussian process.

For the regression problem we use the $R^2$ score, which is given by
$$R^2(y, \hat y) = 1 - \frac{\sum_{i = 1}^{n} (y_i - \hat y_i)^2}{\sum_{i = 1}^{n} (y_i - \bar y_i)^2},$$
where $\bar y_i = \frac 1 n \sum_{i = 1}^n y_i$. Here $y$ is the vector of true answers on the test set, and $\hat y$ is the vector of predicted answers.

For the classification problem we use accuracy score.

In all the experiments we used the squared exponential covariance function. 

\subsection{Inducing input methods and standard methods}
	% \begin{figure}[!t]
% 	\centering
% 	\subfloat{
% 		\scalebox{0.8}{
% 			\input{../../Code/Experiments/plots/inducing_inputs/d1_n500.pgf}
% 		}
% 	}
% 	\subfloat{
% 		\scalebox{0.8}{
% 			\input{../../Code/Experiments/plots/inducing_inputs/d5_n500.pgf}
% 		}
% 	}

% 	\subfloat{
% 		\scalebox{0.8}{
% 			\input{../../Code/Experiments/plots/inducing_inputs/d10_n4000.pgf}
% 		}
% 	}
% 	\subfloat{
% 		\scalebox{0.8}{
% 			\input{../../Code/Experiments/plots/inducing_inputs/abalone.pgf}
% 		}
% 	}
% 	\caption{The dependence between prediction quality and the number of inducing inputs for the regression problem}
% 	\label{ind_points_results}
% \end{figure}

\begin{figure}[!t]
	\centering
	\subfloat{
		\scalebox{0.8}{
			\input{../../Code/Experiments/plots/inducing_inputs/d5_n500.pgf}
		}
	}
	\subfloat{
		\scalebox{0.8}{
			\input{../../Code/Experiments/plots/inducing_inputs/class_d2_n200.pgf}
		}
	}
	\caption{The dependence between the prediction quality and the number of inducing inputs for standard and inducing point methods}
	\label{ind_vs_std}
\end{figure}


We've seen above, that inducing input methods have a much smaller computational complexity than the standard methods for both GP-regression and GP-classification problems. In this section we empirically compare the prediction quality on the test data for these methods with the quality obtained by the standard methods.

We also explore the dependence between the number of inducing points used by the method and the prediction quality.

For the regression problem we use two variations of the \lstinline{vi} method. \lstinline{vi-means} method does not maximize the lower bound with respect to the positions of inducing inputs and just uses the K-Means cluster centers as the positions of inducing inputs. The \lstinline{vi} method on the other hand does optimize for the inducing input positions. \lstinline{full GP} method is the standard GP-regression method, described in section \ref{gp_regression}.

Fig. \ref{ind_vs_std} shows the dependence between the prediction quality and the number of inducing inputs for \lstinline{vi} methods for regression and classification problems.

As we can see, optimization with respect to the positions of inducing inputs does not dramatically increase the quality of predictions in the provided experiment. It does however make the optimization problem that we have to solve much harder. In general, the optimization of inducing input positions does increase the prediction quality, but makes the method much slower. We will thus abandon this method and only use \lstinline{vi-means} in other experiments.

% We can also see from these plots that for sufficient number of inducing inputs the \lstinline{vi} methods reach the predictive quality of the standard methods. 

% In the bottom plots of fig. \ref{ind_points_results}, the dependence between the number of used inducing inputs and the quality of the \lstinline{vi-means} method predictions is shown for two bigger datasets. 

% \begin{figure}[!t]
% 	\centering
% 	\subfloat{
% 		\scalebox{0.8}{
% 			\input{../../Code/Experiments/plots/inducing_inputs/class_d2_n200.pgf}
% 		}
% 	}
% 	\subfloat{
% 		\scalebox{0.8}{
% 			\input{../../Code/Experiments/plots/inducing_inputs/class_german.pgf}
% 		}
% 	}
% 	\caption{The dependence between prediction quality and the number of inducinge inputs for the classification problem}
% 	\label{ind_inputs_class_results}
% \end{figure}

\begin{figure}[!t]
	\centering
	\subfloat{
		\scalebox{0.8}{
			\input{../../Code/Experiments/plots/inducing_inputs/abalone.pgf}
		}
	}
	\subfloat{
		\scalebox{0.8}{
			\input{../../Code/Experiments/plots/inducing_inputs/class_german.pgf}
		}
	}
	\caption{The dependence between prediction quality and the number of inducing inputs for \lstinline{vi-means} and \lstinline{svi-classification} methods}
	\label{ind_inputs_results}
\end{figure}

Fig. \ref{ind_inputs_results} shows the dependence between the prediction quality and the number of inducing inputs for lstinline{vi-means} and \lstinline{svi-classification} methods. As we can see, the prediction quality gets better as the number of inducing inputs grows. However, it's hard to say, how many inducing points one should use in practice. The best answer is probably the biggest amount one can afford to train. 

% In fig. \ref{ind_inputs_class_results} the results of similar experiments for the classification problem are provided. Here we compare the \lstinline{Laplace} method, which was described in section \ref{gp-classification}, and the \lstinline{svi-classification} method, which was described in section \ref{svi_classification}.

% As we can see, on the chosen dataset the \lstinline{svi-classification} method can not reach the predictive quality of the \lstinline{Laplace} method even for reasonably big values of $m$. However, the time consumption of the \lstinline{Laplace} method does not allow to use it even for moderate~problems. 




\subsection{SVI method variations}
	\begin{figure}[t!]
	\centering

	\subfloat{
		\scalebox{0.75}{
			\input{../../Code/Experiments/Plots/svi_variations/small_generated.pgf}
		}
	}
	\subfloat{
		\scalebox{0.75}{
    		\input{../../Code/Experiments/Plots/svi_variations/small_real.pgf}
		}
	}
	\vspace{0.1cm}
	\subfloat{
		\scalebox{0.75}{
			\input{../../Code/Experiments/Plots/svi_variations/medium_generated.pgf}
		}
	}
	\subfloat{
		\scalebox{0.75}{
    		\input{../../Code/Experiments/Plots/svi_variations/medium_real.pgf}
		}
	}
	\caption{\lstinline{svi} methods' performance on small and medium datasets}
	\label{svi_results}
\end{figure}
In this section we compare several variations of the \lstinline{svi} method for the regression problem.

The first variation is denoted by \lstinline{svi-natural}. It is the method as it was proposed in \cite{BigData}. It uses stochastic gradient descent with natural gradients for minimizing the ELBO with respect to the variational parameters, and usual gradients with respect to kernel hyper-parameters.

The methods \lstinline{svi-L-BFGS-B} and \lstinline{svi-FG} use the same lower bound (\ref{svi_elbo}) and optimize it with deterministic optimization methods L-BFGS-B and projected gradient respectively. We use the bound-constrained optimization methods, because the hyper-parameters of the squared exponential kernel must be positive.

We can not use the natural gradients in this setting, because they are not necessarily a descent direction and can't be used by L-BFGS-B or gradient descent. Thus, we use usual gradients with respect to variational parameters $\mu$ and $\Sigma$ for these methods. However, the matrix $\Sigma$ has to be symmetric and positive definite and we have to ensure that our optimization updates maintain these properties. In order to avoid complex constrained optimization problems, we use Cholesky decomposition of $\Sigma$ and optimize the bound with respect to the Cholesky factor $L_\Sigma$ of $\Sigma$. This allows us to solve a simpler bound-constrained problem instead of a general constrained optimization problem.

Finally, the \lstinline{svi-SAG} uses stochastic average gradient method to minimize the ELBO. This method also uses Cholesky factorization and usual gradients instead of natural for the same reasons. For more information about SAG method see \cite{SAG}.


The results on small and medium datasets are shown in fig. \ref{svi_results}.

As we can see, on these moderate problems using stochastic optimization does not give any advantages against the L-BFGS-B method. However, using the natural gradients allows the stochastic gradient descent method to beat SAG. For these reasons we will only use the \lstinline{svi-natural} and \lstinline{svi-L-BFGS-B} methods in the comparison with the \lstinline{vi-means} method.
\subsection{VI method variations}
	\begin{figure}[t!]
	\centering
	\subfloat{
		\scalebox{0.75}{
			\input{../../Code/Experiments/Plots/vi_variations/small_real.pgf}
		}
	}
	\subfloat{
		\scalebox{0.75}{
    		\input{../../Code/Experiments/Plots/vi_variations/medium_real.pgf}
		}
	}

	% \subfloat{
	% 	\scalebox{0.75}{
	% 		\input{../../Code/Experiments/Plots/vi_variations/big_real.pgf}
	% 	}
	% }
	% \subfloat{
	% 	\scalebox{0.75}{
	% 		\input{../../Code/Experiments/Plots/vi_variations/huge_real.pgf}
	% 	}
	% }
	\caption{ \lstinline{vi} method variations on different datasets}
	\label{vi_results}
\end{figure}

In this section we compare two optimization methods for the \lstinline{vi-means} method.

The first variation is denoted by \lstinline{Projected Newton}. It uses projected Newton method for minimizing the ELBO (\ref{titsias_elbo}). The second variation is denoted by \lstinline{means-L-BFGS-B} and uses L-BFGS-B optimization method.

\lstinline{Projected Newton} method uses finite-difference approximation of the hessian. It also makes hessian-correction in order to make it symmetric and positive-definite. The optimization method itself makes a Newton step and then projects the result to the feasible set in the metric, determined by the hessian. For more information about the method see for example \cite{ProjNewton}.

The time complexity of one iteration for both projected Newton method and L-BFGS-B is $\bigO(nm^2)$. In the projected Newton method we have to compute the hessian matrix of the ELBO with respect to covariance hyper-parameters. In case of squared exponential covariance function the time, needed to compute the hessian, is twice the time, needed to compute the gradient.

The results are provided in fig. \ref{vi_results}. In the provided experiments projected Newton method beats L-BFGS-B. However, the results are close and on different datasets L-BFGS-B beats projected Newton. We need to perform further experiments in order to find out whether using second order optimization provides benefits in the \lstinline{vi} method.

In further experiments we use the L-BFGS-B method because it was more stable in general in our experiments.
\subsection{Comparison of VI and SVI methods}
	In this section we compare the \lstinline{vi-means} method (with L-BFGS-B optimization method) with \lstinline{svi-L-BFGS-B} and \lstinline{svi-natural}. We've described these methods above. 

We've seen, that the computational complexity of one epoch of the \lstinline{vi-means} method is better method than it is for the \lstinline{svi-natural} and \lstinline{svi-L-BFGS-B}. However, the \lstinline{svi-natural} method uses stochastic optimization, which might lead to faster convergence in big data problems.

\begin{figure}[!h]
	\centering
	\subfloat{
		\scalebox{0.75}{
	    	\input{../../Code/Experiments/Plots/vi_vs_svi/medium_real.pgf}
		}
	}
	\subfloat{
		\scalebox{0.73}{
			\input{../../Code/Experiments/Plots/vi_vs_svi/1e5_sg_lbfgs.pgf}
		}
	}
	\caption{\lstinline{vi} and \lstinline{svi} methods comparison}
	\label{visvi_results}
\end{figure}

Fig. \ref{visvi_results} shows the results of the experimental comparison of \lstinline{vi} and \lstinline{svi} methods. In the first experiment of these two we didn't run the \lstinline{svi-natural}, because we've already compared it to the \lstinline{svi-L-BFGS-B} method on this exact dataset and it proved to be worse (see fig. \ref{svi_results}).

We can see, that \lstinline{svi-natural} performs slightly better, then the deterministic \lstinline{svi-L-BFGS-B}, but can't beat \lstinline{vi-means}. The reason for that is that the optimization problem for the \lstinline{svi} method is much harder then the one, solved by the \lstinline{vi} method. Indeed, for $m = 1000$ and squared exponential kernel we have about $5 \cdot 10^5$ optimization parameters for \lstinline{svi} methods and just $3$ parameters for the \lstinline{vi} method.
